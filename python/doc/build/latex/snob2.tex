%% Generated by Sphinx.
\def\sphinxdocclass{report}
\documentclass[letterpaper,10pt,english]{sphinxmanual}
\ifdefined\pdfpxdimen
   \let\sphinxpxdimen\pdfpxdimen\else\newdimen\sphinxpxdimen
\fi \sphinxpxdimen=.75bp\relax
\ifdefined\pdfimageresolution
    \pdfimageresolution= \numexpr \dimexpr1in\relax/\sphinxpxdimen\relax
\fi
%% let collapsible pdf bookmarks panel have high depth per default
\PassOptionsToPackage{bookmarksdepth=5}{hyperref}

\PassOptionsToPackage{warn}{textcomp}
\usepackage[utf8]{inputenc}
\ifdefined\DeclareUnicodeCharacter
% support both utf8 and utf8x syntaxes
  \ifdefined\DeclareUnicodeCharacterAsOptional
    \def\sphinxDUC#1{\DeclareUnicodeCharacter{"#1}}
  \else
    \let\sphinxDUC\DeclareUnicodeCharacter
  \fi
  \sphinxDUC{00A0}{\nobreakspace}
  \sphinxDUC{2500}{\sphinxunichar{2500}}
  \sphinxDUC{2502}{\sphinxunichar{2502}}
  \sphinxDUC{2514}{\sphinxunichar{2514}}
  \sphinxDUC{251C}{\sphinxunichar{251C}}
  \sphinxDUC{2572}{\textbackslash}
\fi
\usepackage{cmap}
\usepackage[T1]{fontenc}
\usepackage{amsmath,amssymb,amstext}
\usepackage{babel}



\usepackage{tgtermes}
\usepackage{tgheros}
\renewcommand{\ttdefault}{txtt}



\usepackage[Bjarne]{fncychap}
\usepackage{sphinx}

\fvset{fontsize=auto}
\usepackage{geometry}


% Include hyperref last.
\usepackage{hyperref}
% Fix anchor placement for figures with captions.
\usepackage{hypcap}% it must be loaded after hyperref.
% Set up styles of URL: it should be placed after hyperref.
\urlstyle{same}

\addto\captionsenglish{\renewcommand{\contentsname}{Contents:}}

\usepackage{sphinxmessages}
\setcounter{tocdepth}{1}



\title{Snob2}
\date{Oct 31, 2021}
\release{}
\author{Risi Kondor}
\newcommand{\sphinxlogo}{\vbox{}}
\renewcommand{\releasename}{}
\makeindex
\begin{document}

\pagestyle{empty}
\sphinxmaketitle
\pagestyle{plain}
\sphinxtableofcontents
\pagestyle{normal}
\phantomsection\label{\detokenize{index::doc}}


\begin{DUlineblock}{0em}
\item[] 
\end{DUlineblock}

\sphinxAtStartPar
Snob2 is a C++ package with a Python interface
for computing the representations of the symmetric group \(\mathbb{S}_n\) and
computing fast Fourier transforms on \(\mathbb{S}_n\).

\sphinxAtStartPar
Snob2 is built on the \sphinxstylestrong{cnine} library which can be downloaded from \sphinxurl{https://github.com/risi-kondor/cnine}.
GPU functionality for the library is not yet available but is under development.

\sphinxAtStartPar
Snob2 is written by Risi Kondor at the University of Chicago and is released under the
\sphinxhref{https://www.mozilla.org/en-US/MPL/2.0/}{Mozilla public license v.2.0}.

\sphinxAtStartPar
This document provides documentation for Snob2’s Python interface. Not all features in the C++ library
are available through this interface. The documentation of the C++ API can be found in pfd format
in the package’s \sphinxcode{\sphinxupquote{doc}} directory.


\chapter{Features}
\label{\detokenize{index:features}}\begin{itemize}
\item {} 
\sphinxAtStartPar
Custom classes for combinatorial objects such as integer partitions and Young tableau implemented in C++
for efficiency.

\item {} 
\sphinxAtStartPar
Classes for the symmetric group, conjugacy classes, quotient spaces,
characters and irreducible representations of \(\mathbb{S}_n\).

\item {} 
\sphinxAtStartPar
Classes for functions on \(\mathbb{S}_n\) and quotient spaces of \(\mathbb{S}_n\).

\item {} 
\sphinxAtStartPar
Classes implementing Clausen’s FFT algorithm for the forward and backward Fourier transform on
\(\mathbb{S}_n\) and its quotient spaces.

\end{itemize}


\chapter{Installation}
\label{\detokenize{index:installation}}
\sphinxAtStartPar
Installing Snob2 requires the following:
\begin{enumerate}
\sphinxsetlistlabels{\arabic}{enumi}{enumii}{}{.}%
\item {} 
\sphinxAtStartPar
A C++ installation with C++11 or higher

\item {} 
\sphinxAtStartPar
Python

\end{enumerate}

\sphinxAtStartPar
To install Snob2 follow these steps:
\begin{enumerate}
\sphinxsetlistlabels{\arabic}{enumi}{enumii}{}{.}%
\item {} 
\sphinxAtStartPar
Download the \sphinxhref{https://github.com/risi-kondor/cnine}{cnine} and
\sphinxhref{https://github.com/risi-kondor/Snob2}{Snob2} packages.

\item {} 
\sphinxAtStartPar
Edit the file \sphinxcode{\sphinxupquote{options.txt}}, in particular, make sure that \sphinxcode{\sphinxupquote{CNINE\_ROOT}} points to the root of
the \sphinxstylestrong{cnine} package on your system.

\item {} 
\sphinxAtStartPar
Run \sphinxcode{\sphinxupquote{python setup.sty install}} in the \sphinxcode{\sphinxupquote{pytorch}} directory to compile the package and install it on your
system.

\end{enumerate}

\sphinxAtStartPar
To use Snob2, issue the command \sphinxcode{\sphinxupquote{import Snob2}} in Python. This loads to the \sphinxtitleref{Snob2} module and initializes
the various static datastructures used by the package.


\chapter{Design}
\label{\detokenize{index:design}}
\sphinxAtStartPar
The representation theory of \(\mathbb{S}_n\) involves some data structures that are relatively
expensive to compute but only needed to be computed once. Snob2’s backend automatically caches these data.
For example, the class \sphinxcode{\sphinxupquote{IntegerPartitions}} returns all integer partitions of an integer \(n\).
The first time that an \sphinxcode{\sphinxupquote{IntegerPartitions}} object is created for a given value of \(n\), Snob2
constructs the integer partitions from the integer partitions of \(m<n\) and stores the result
in a static object so that on subsequent calls the same process does not need to be repeated.

\begin{DUlineblock}{0em}
\item[] 
\end{DUlineblock}


\chapter{Classes}
\label{\detokenize{index:classes}}

\section{Combinatorial classes}
\label{\detokenize{index:combinatorial-classes}}
\sphinxAtStartPar
Snob provides specialized classes to represent various combinatorial objects involved in the
representation theory of \(\mathbb{S}_n\).


\subsection{Permutations}
\label{\detokenize{index:permutations}}
\sphinxAtStartPar
A permutation \(\pi\) of n is a bijective map \(\{1,2,\ldots,n\}\to\{1,2,\ldots,n\}\).

\begin{sphinxVerbatim}[commandchars=\\\{\}]
\PYG{g+gp}{\PYGZgt{}\PYGZgt{}\PYGZgt{} }\PYG{n}{pi}\PYG{o}{=}\PYG{n}{Snob2}\PYG{o}{.}\PYG{n}{Permutation}\PYG{p}{(}\PYG{p}{[}\PYG{l+m+mi}{2}\PYG{p}{,}\PYG{l+m+mi}{3}\PYG{p}{,}\PYG{l+m+mi}{1}\PYG{p}{,}\PYG{l+m+mi}{5}\PYG{p}{,}\PYG{l+m+mi}{4}\PYG{p}{]}\PYG{p}{)}
\PYG{g+gp}{\PYGZgt{}\PYGZgt{}\PYGZgt{} }\PYG{n+nb}{print}\PYG{p}{(}\PYG{n}{pi}\PYG{p}{)}
\PYG{g+go}{[ 2 3 1 5 4 ]}
\PYG{g+gp}{\PYGZgt{}\PYGZgt{}\PYGZgt{} }\PYG{n}{pi}\PYG{p}{[}\PYG{l+m+mi}{3}\PYG{p}{]}
\PYG{g+go}{1}
\end{sphinxVerbatim}

\sphinxAtStartPar
The product of two permutations \(\tau\) and \(\pi\) is the permutation corresponding
to the composition of maps \(\tau\circ\pi\).

\begin{sphinxVerbatim}[commandchars=\\\{\}]
\PYG{g+gp}{\PYGZgt{}\PYGZgt{}\PYGZgt{} }\PYG{n}{tau}\PYG{o}{=}\PYG{n}{Snob2}\PYG{o}{.}\PYG{n}{Permutation}\PYG{p}{(}\PYG{p}{[}\PYG{l+m+mi}{2}\PYG{p}{,}\PYG{l+m+mi}{1}\PYG{p}{,}\PYG{l+m+mi}{3}\PYG{p}{,}\PYG{l+m+mi}{4}\PYG{p}{,}\PYG{l+m+mi}{5}\PYG{p}{]}\PYG{p}{)}
\PYG{g+gp}{\PYGZgt{}\PYGZgt{}\PYGZgt{} }\PYG{n+nb}{print}\PYG{p}{(}\PYG{n}{tau}\PYG{o}{*}\PYG{n}{pi}\PYG{p}{)}
\PYG{g+go}{[ 1 3 2 5 4 ]}
\end{sphinxVerbatim}

\sphinxAtStartPar
The \sphinxtitleref{inv} method returns the inverse of a permutation.

\begin{sphinxVerbatim}[commandchars=\\\{\}]
\PYG{g+gp}{\PYGZgt{}\PYGZgt{}\PYGZgt{} }\PYG{n+nb}{print}\PYG{p}{(}\PYG{n}{pi}\PYG{o}{.}\PYG{n}{inv}\PYG{p}{(}\PYG{p}{)}\PYG{p}{)}
\PYG{g+go}{[ 3 1 2 5 4 ]}
\end{sphinxVerbatim}


\subsection{Integer partitions}
\label{\detokenize{index:integer-partitions}}
\sphinxAtStartPar
An \sphinxstyleemphasis{integer partition} of a positive integer \(n\) is a vector of positive integers
\(\lambda=(\lambda_1,\ldots,\lambda_k)\) such that \(\sum_{i=1}^k \lambda_i=n\) and
\(\lambda_1\geq \lambda_2\geq \ldots\geq\lambda_k\).
The \sphinxcode{\sphinxupquote{IntegerPartition}} class represents such vectors.

\begin{sphinxVerbatim}[commandchars=\\\{\}]
\PYG{g+gp}{\PYGZgt{}\PYGZgt{}\PYGZgt{} }\PYG{n}{a}\PYG{o}{=}\PYG{n}{Snob2}\PYG{o}{.}\PYG{n}{IntegerPartition}\PYG{p}{(}\PYG{p}{[}\PYG{l+m+mi}{3}\PYG{p}{,}\PYG{l+m+mi}{2}\PYG{p}{,}\PYG{l+m+mi}{1}\PYG{p}{]}\PYG{p}{)}
\PYG{g+gp}{\PYGZgt{}\PYGZgt{}\PYGZgt{} }\PYG{k}{print}\PYG{p}{(}\PYG{n}{a}\PYG{p}{)}
\PYG{g+go}{[3,2,1]}
\PYG{g+gp}{\PYGZgt{}\PYGZgt{}\PYGZgt{} }\PYG{k}{print}\PYG{p}{(}\PYG{n}{a}\PYG{p}{[}\PYG{l+m+mi}{1}\PYG{p}{]}\PYG{p}{)}
\PYG{g+go}{2}
\end{sphinxVerbatim}

\sphinxAtStartPar
The \sphinxcode{\sphinxupquote{IntegerPartitions}} class returns an object that contains \sphinxstyleemphasis{all} integer partitions of math:\sphinxtitleref{n}.

\begin{sphinxVerbatim}[commandchars=\\\{\}]
\PYG{g+gp}{\PYGZgt{}\PYGZgt{}\PYGZgt{} }\PYG{n}{Lambda}\PYG{o}{=}\PYG{n}{Snob2}\PYG{o}{.}\PYG{n}{IntegerPartitions}\PYG{p}{(}\PYG{l+m+mi}{5}\PYG{p}{)}
\PYG{g+gp}{\PYGZgt{}\PYGZgt{}\PYGZgt{} }\PYG{k}{for} \PYG{n}{i} \PYG{o+ow}{in} \PYG{n+nb}{range}\PYG{p}{(}\PYG{n+nb}{len}\PYG{p}{(}\PYG{n}{Lambda}\PYG{p}{)}\PYG{p}{)}\PYG{p}{:}
\PYG{g+gp}{... }     \PYG{k}{print}\PYG{p}{(}\PYG{n}{Lambda}\PYG{p}{[}\PYG{n}{i}\PYG{p}{]}\PYG{p}{)}
\PYG{g+gp}{...}
\PYG{g+go}{[5]}
\PYG{g+go}{[4,1]}
\PYG{g+go}{[3,2]}
\PYG{g+go}{[3,1,1]}
\PYG{g+go}{[2,2,1]}
\PYG{g+go}{[2,1,1,1]}
\PYG{g+go}{[1,1,1,1,1]}
\end{sphinxVerbatim}


\subsection{Young tableau}
\label{\detokenize{index:young-tableau}}
\sphinxAtStartPar
A Young tableau is a Young diagram whose cells are filled with integers. The
default Young tableau of a given shape is the one where the numbers \(1,2,\ldots,n\)
appear sequentially.

\begin{sphinxVerbatim}[commandchars=\\\{\}]
\PYG{g+gp}{\PYGZgt{}\PYGZgt{}\PYGZgt{} }\PYG{n}{T}\PYG{o}{=}\PYG{n}{Snob2}\PYG{o}{.}\PYG{n}{YoungTableau}\PYG{p}{(}\PYG{n}{Snob2}\PYG{o}{.}\PYG{n}{IntegerPartition}\PYG{p}{(}\PYG{p}{[}\PYG{l+m+mi}{3}\PYG{p}{,}\PYG{l+m+mi}{2}\PYG{p}{,}\PYG{l+m+mi}{1}\PYG{p}{]}\PYG{p}{)}\PYG{p}{)}
\PYG{g+gp}{\PYGZgt{}\PYGZgt{}\PYGZgt{} }\PYG{k}{print}\PYG{p}{(}\PYG{n}{T}\PYG{p}{)}
\PYG{g+go}{1 2 3}
\PYG{g+go}{4 5}
\PYG{g+go}{6}
\PYG{g+gp}{\PYGZgt{}\PYGZgt{}\PYGZgt{} }\PYG{k}{print}\PYG{p}{(}\PYG{n}{T}\PYG{o}{.}\PYG{n}{shape}\PYG{p}{(}\PYG{p}{)}\PYG{p}{)}
\PYG{g+go}{[3,2,1]}
\PYG{g+go}{\PYGZgt{}\PYGZgt{}\PYGZgt{}}
\end{sphinxVerbatim}

\sphinxAtStartPar
A \sphinxstyleemphasis{standard Young tableau} is a Young tableau filled with the numbers \(1,2,\ldots,n\)
in such a way that in any row the numbers increase from left to right and in any
column the numbers increase top to bottom.
The \sphinxcode{\sphinxupquote{StandardYoungTableaux}} class returns an object that contains \sphinxstyleemphasis{all} standard Young tableaux
of a given shape.

\begin{sphinxVerbatim}[commandchars=\\\{\}]
\PYG{g+gp}{\PYGZgt{}\PYGZgt{}\PYGZgt{} }\PYG{n}{lamb}\PYG{o}{=}\PYG{n}{Snob2}\PYG{o}{.}\PYG{n}{IntegerPartition}\PYG{p}{(}\PYG{p}{[}\PYG{l+m+mi}{3}\PYG{p}{,}\PYG{l+m+mi}{2}\PYG{p}{]}\PYG{p}{)}
\PYG{g+gp}{\PYGZgt{}\PYGZgt{}\PYGZgt{} }\PYG{n}{T}\PYG{o}{=}\PYG{n}{Snob2}\PYG{o}{.}\PYG{n}{StandardYoungTableaux}\PYG{p}{(}\PYG{n}{lamb}\PYG{p}{)}
\PYG{g+gp}{\PYGZgt{}\PYGZgt{}\PYGZgt{} }\PYG{k}{for} \PYG{n}{i} \PYG{o+ow}{in} \PYG{n+nb}{range}\PYG{p}{(}\PYG{n+nb}{len}\PYG{p}{(}\PYG{n}{T}\PYG{p}{)}\PYG{p}{)}\PYG{p}{:}
\PYG{g+gp}{... }    \PYG{k}{print}\PYG{p}{(}\PYG{n}{T}\PYG{p}{[}\PYG{n}{i}\PYG{p}{]}\PYG{p}{)}
\PYG{g+gp}{...}
\PYG{g+go}{1 2 3}
\PYG{g+go}{4 5}

\PYG{g+go}{1 2 4}
\PYG{g+go}{3 5}

\PYG{g+go}{1 3 4}
\PYG{g+go}{2 5}

\PYG{g+go}{1 2 5}
\PYG{g+go}{3 4}

\PYG{g+go}{1 3 5}
\PYG{g+go}{2 4}
\end{sphinxVerbatim}

\begin{DUlineblock}{0em}
\item[] 
\end{DUlineblock}


\section{Symmetric group classes}
\label{\detokenize{index:symmetric-group-classes}}
\sphinxAtStartPar
The symmetric group \(\mathbb{S}_n\), represented by the class \sphinxcode{\sphinxupquote{Sn}},
is the group of all permutations of \(\{1,2,\ldots,n\}\).

\begin{sphinxVerbatim}[commandchars=\\\{\}]
\PYG{g+gp}{\PYGZgt{}\PYGZgt{}\PYGZgt{} }\PYG{n}{G}\PYG{o}{=}\PYG{n}{Snob2}\PYG{o}{.}\PYG{n}{Sn}\PYG{p}{(}\PYG{l+m+mi}{4}\PYG{p}{)}
\PYG{g+gp}{\PYGZgt{}\PYGZgt{}\PYGZgt{} }\PYG{k}{for} \PYG{n}{i} \PYG{o+ow}{in} \PYG{n+nb}{range}\PYG{p}{(}\PYG{n+nb}{len}\PYG{p}{(}\PYG{n}{G}\PYG{p}{)}\PYG{p}{)}\PYG{p}{:}
\PYG{g+gp}{... }    \PYG{k}{print}\PYG{p}{(}\PYG{n}{G}\PYG{o}{.}\PYG{n}{element}\PYG{p}{(}\PYG{n}{i}\PYG{p}{)}\PYG{p}{)}
\PYG{g+gp}{...}
\PYG{g+go}{[ 1 2 3 4 ]}
\PYG{g+go}{[ 2 1 3 4 ]}
\PYG{g+go}{[ 1 3 2 4 ]}
\PYG{g+go}{[ 2 3 1 4 ]}
\PYG{g+go}{[ 3 1 2 4 ]}
\PYG{g+go}{[ 3 2 1 4 ]}
\PYG{g+go}{[ 1 2 4 3 ]}
\PYG{g+go}{[ 2 1 4 3 ]}
\PYG{g+go}{[ 1 3 4 2 ]}
\PYG{g+go}{[ 2 3 4 1 ]}
\PYG{g+go}{[ 3 1 4 2 ]}
\PYG{g+go}{[ 3 2 4 1 ]}
\PYG{g+go}{[ 1 4 2 3 ]}
\PYG{g+go}{[ 2 4 1 3 ]}
\PYG{g+go}{[ 1 4 3 2 ]}
\PYG{g+go}{[ 2 4 3 1 ]}
\PYG{g+go}{[ 3 4 1 2 ]}
\PYG{g+go}{[ 3 4 2 1 ]}
\PYG{g+go}{[ 4 1 2 3 ]}
\PYG{g+go}{[ 4 2 1 3 ]}
\PYG{g+go}{[ 4 1 3 2 ]}
\PYG{g+go}{[ 4 2 3 1 ]}
\PYG{g+go}{[ 4 3 1 2 ]}
\PYG{g+go}{[ 4 3 2 1 ]}
\end{sphinxVerbatim}


\subsection{Group elements}
\label{\detokenize{index:group-elements}}
\sphinxAtStartPar
The group elements of \(\mathbb{S}_n\) are of type \sphinxcode{\sphinxupquote{SnElement}}, which have the same methods as the
\sphinxcode{\sphinxupquote{Permutation}} class. Group elements are listed in a specific reverse insertion sort order that fits
the structure of \(\sigma\mathbb{S}_m\) cosets and hence is well adapted to Clausen\sphinxhyphen{}type
fast Fourier transforms on \(\mathbb{S}_n\). The following shows how to extract the i’th
group element and how to get the index of a particular group element.

\begin{sphinxVerbatim}[commandchars=\\\{\}]
\PYG{g+gp}{\PYGZgt{}\PYGZgt{}\PYGZgt{} }\PYG{n}{pi}\PYG{o}{=}\PYG{n}{G}\PYG{p}{[}\PYG{l+m+mi}{17}\PYG{p}{]}
\PYG{g+gp}{\PYGZgt{}\PYGZgt{}\PYGZgt{} }\PYG{k}{print}\PYG{p}{(}\PYG{n}{pi}\PYG{p}{)}
\PYG{g+go}{[ 3 4 2 1 ]}
\PYG{g+gp}{\PYGZgt{}\PYGZgt{}\PYGZgt{} }\PYG{k}{print}\PYG{p}{(}\PYG{n}{G}\PYG{o}{.}\PYG{n}{index}\PYG{p}{(}\PYG{n}{pi}\PYG{p}{)}\PYG{p}{)}
\PYG{g+go}{17}
\end{sphinxVerbatim}


\subsection{Conjugacy classes}
\label{\detokenize{index:conjugacy-classes}}
\sphinxAtStartPar
The conjugacy classes of \(\mathbb{S}_n\) are in bijection with the integer partitions of \(n\).
Snob2 has a separate class \sphinxcode{\sphinxupquote{SnCClass}} to represent conjugacy classes. \sphinxcode{\sphinxupquote{SnCClass}} objects can
be constructed either from the group object or directly from an integer partition.

\begin{sphinxVerbatim}[commandchars=\\\{\}]
\PYG{g+gp}{\PYGZgt{}\PYGZgt{}\PYGZgt{} }\PYG{n}{G}\PYG{o}{=}\PYG{n}{Snob2}\PYG{o}{.}\PYG{n}{Sn}\PYG{p}{(}\PYG{l+m+mi}{5}\PYG{p}{)}
\PYG{g+gp}{\PYGZgt{}\PYGZgt{}\PYGZgt{} }\PYG{n}{mu}\PYG{o}{=}\PYG{n}{Snob2}\PYG{o}{.}\PYG{n}{IntegerPartition}\PYG{p}{(}\PYG{p}{[}\PYG{l+m+mi}{3}\PYG{p}{,}\PYG{l+m+mi}{2}\PYG{p}{]}\PYG{p}{)}
\PYG{g+gp}{\PYGZgt{}\PYGZgt{}\PYGZgt{} }\PYG{n}{cc}\PYG{o}{=}\PYG{n}{G}\PYG{o}{.}\PYG{n}{cclass}\PYG{p}{(}\PYG{n}{mu}\PYG{p}{)}
\PYG{g+gp}{\PYGZgt{}\PYGZgt{}\PYGZgt{} }\PYG{k}{print}\PYG{p}{(}\PYG{n}{cc}\PYG{p}{)}
\PYG{g+go}{SnCClass[3,2]}
\end{sphinxVerbatim}

\sphinxAtStartPar
The conjugacy classes are ordered according to majorization order of their integer partitions.
The \sphinxcode{\sphinxupquote{Sn.index}} method returns the index of a given conjugacy class.

\begin{sphinxVerbatim}[commandchars=\\\{\}]
\PYG{g+gp}{\PYGZgt{}\PYGZgt{}\PYGZgt{} }\PYG{n}{G}\PYG{o}{.}\PYG{n}{index}\PYG{p}{(}\PYG{n}{cc}\PYG{p}{)}
\PYG{g+go}{2}
\end{sphinxVerbatim}


\subsection{Characters}
\label{\detokenize{index:characters}}
\sphinxAtStartPar
The characters of \(\mathbb{S}_n\) are also indexed by the integer partitions of \(n\)
and can be accessed through the \sphinxcode{\sphinxupquote{character}} method of \sphinxcode{\sphinxupquote{Sn}}.

\begin{sphinxVerbatim}[commandchars=\\\{\}]
\PYG{g+gp}{\PYGZgt{}\PYGZgt{}\PYGZgt{} }\PYG{n}{G}\PYG{o}{=}\PYG{n}{Snob2}\PYG{o}{.}\PYG{n}{Sn}\PYG{p}{(}\PYG{l+m+mi}{5}\PYG{p}{)}
\PYG{g+gp}{\PYGZgt{}\PYGZgt{}\PYGZgt{} }\PYG{n}{lambd}\PYG{o}{=}\PYG{n}{Snob2}\PYG{o}{.}\PYG{n}{IntegerPartition}\PYG{p}{(}\PYG{p}{[}\PYG{l+m+mi}{3}\PYG{p}{,}\PYG{l+m+mi}{2}\PYG{p}{]}\PYG{p}{)}
\PYG{g+gp}{\PYGZgt{}\PYGZgt{}\PYGZgt{} }\PYG{n}{chi}\PYG{o}{=}\PYG{n}{G}\PYG{o}{.}\PYG{n}{character}\PYG{p}{(}\PYG{n}{lambd}\PYG{p}{)}
\PYG{g+gp}{\PYGZgt{}\PYGZgt{}\PYGZgt{} }\PYG{k}{print}\PYG{p}{(}\PYG{n}{chi}\PYG{p}{)}
\PYG{g+go}{SnCClass[5] : 0}
\PYG{g+go}{SnCClass[4,1] : \PYGZhy{}1}
\PYG{g+go}{SnCClass[3,2] : 1}
\PYG{g+go}{SnCClass[3,1,1] : \PYGZhy{}1}
\PYG{g+go}{SnCClass[2,2,1] : 1}
\PYG{g+go}{SnCClass[2,1,1,1] : 1}
\PYG{g+go}{SnCClass[1,1,1,1,1] : 5}
\end{sphinxVerbatim}


\subsection{Irreducible representations}
\label{\detokenize{index:irreducible-representations}}
\sphinxAtStartPar
The irreducible representations (irreps) of \(\mathbb{S}_n\) are captured by \sphinxcode{\sphinxupquote{SnIrrep}} objects. For a
given integer partition \(\lambda\) of n, the corresponding irrep can be constructed from
the group object or directly from the integer partition.

\begin{sphinxVerbatim}[commandchars=\\\{\}]
\PYG{g+gp}{\PYGZgt{}\PYGZgt{}\PYGZgt{} }\PYG{n}{lambd}\PYG{o}{=}\PYG{n}{Snob2}\PYG{o}{.}\PYG{n}{IntegerPartition}\PYG{p}{(}\PYG{p}{[}\PYG{l+m+mi}{3}\PYG{p}{,}\PYG{l+m+mi}{1}\PYG{p}{]}\PYG{p}{)}
\PYG{g+gp}{\PYGZgt{}\PYGZgt{}\PYGZgt{} }\PYG{n}{rho}\PYG{o}{=}\PYG{n}{G}\PYG{o}{.}\PYG{n}{irrep}\PYG{p}{(}\PYG{n}{lambd}\PYG{p}{)}
\PYG{g+gp}{\PYGZgt{}\PYGZgt{}\PYGZgt{} }\PYG{k}{print}\PYG{p}{(}\PYG{n}{rho}\PYG{p}{)}
\PYG{g+go}{SnIrrep([3,1])}

\PYG{g+gp}{\PYGZgt{}\PYGZgt{}\PYGZgt{} }\PYG{n}{lambd}\PYG{o}{=}\PYG{n}{Snob2}\PYG{o}{.}\PYG{n}{IntegerPartition}\PYG{p}{(}\PYG{p}{[}\PYG{l+m+mi}{3}\PYG{p}{,}\PYG{l+m+mi}{1}\PYG{p}{]}\PYG{p}{)}
\PYG{g+gp}{\PYGZgt{}\PYGZgt{}\PYGZgt{} }\PYG{n}{rho}\PYG{o}{=}\PYG{n}{Snob2}\PYG{o}{.}\PYG{n}{SnIrrep}\PYG{p}{(}\PYG{n}{lambd}\PYG{p}{)}
\PYG{g+gp}{\PYGZgt{}\PYGZgt{}\PYGZgt{} }\PYG{k}{print}\PYG{p}{(}\PYG{n}{rho}\PYG{p}{)}
\PYG{g+go}{SnIrrep([3,1])}
\end{sphinxVerbatim}

\sphinxAtStartPar
The dimension of the irrep is accessible through the \sphinxtitleref{get\_dim()} method.

\begin{sphinxVerbatim}[commandchars=\\\{\}]
\PYG{g+gp}{\PYGZgt{}\PYGZgt{}\PYGZgt{} }\PYG{k}{print}\PYG{p}{(}\PYG{n}{rho}\PYG{o}{.}\PYG{n}{get\PYGZus{}dim}\PYG{p}{(}\PYG{p}{)}\PYG{p}{)}
\PYG{g+go}{3}
\end{sphinxVerbatim}

\sphinxAtStartPar
All irreps in Snob2 are expressed in Young’s orthogonal representation. The representation matrices
are easy to access.

\begin{sphinxVerbatim}[commandchars=\\\{\}]
\PYG{g+gp}{\PYGZgt{}\PYGZgt{}\PYGZgt{} }\PYG{n}{pi}\PYG{o}{=}\PYG{n}{Snob2}\PYG{o}{.}\PYG{n}{SnElement}\PYG{p}{(}\PYG{p}{[}\PYG{l+m+mi}{3}\PYG{p}{,}\PYG{l+m+mi}{2}\PYG{p}{,}\PYG{l+m+mi}{1}\PYG{p}{,}\PYG{l+m+mi}{4}\PYG{p}{]}\PYG{p}{)}
\PYG{g+gp}{\PYGZgt{}\PYGZgt{}\PYGZgt{} }\PYG{k}{print}\PYG{p}{(}\PYG{n}{rho}\PYG{p}{[}\PYG{n}{pi}\PYG{p}{]}\PYG{p}{)}
\PYG{g+go}{[ 1 0 0 ]}
\PYG{g+go}{[ \PYGZhy{}0 \PYGZhy{}0.5 \PYGZhy{}0.866025 ]}
\PYG{g+go}{[ \PYGZhy{}0 \PYGZhy{}0.866025 0.5 ]}
\end{sphinxVerbatim}


\subsection{Sn types}
\label{\detokenize{index:sn-types}}
\sphinxAtStartPar
The \sphinxstyleemphasis{type} of a representation is an associative list of integer partitions and associated multiplicities
describing what irreps a particular representation is composed of.
The following shows how to set up an \sphinxcode{\sphinxupquote{SnType}} object.

\begin{sphinxVerbatim}[commandchars=\\\{\}]
\PYG{g+gp}{\PYGZgt{}\PYGZgt{}\PYGZgt{} }\PYG{n}{tau}\PYG{o}{=}\PYG{n}{Snob2}\PYG{o}{.}\PYG{n}{SnType}\PYG{p}{(}\PYG{n}{Snob2}\PYG{o}{.}\PYG{n}{IntegerPartition}\PYG{p}{(}\PYG{p}{[}\PYG{l+m+mi}{4}\PYG{p}{,}\PYG{l+m+mi}{1}\PYG{p}{]}\PYG{p}{)}\PYG{p}{,}\PYG{l+m+mi}{2}\PYG{p}{)}
\PYG{g+gp}{\PYGZgt{}\PYGZgt{}\PYGZgt{} }\PYG{n}{tau}\PYG{o}{.}\PYG{n}{set}\PYG{p}{(}\PYG{n}{Snob2}\PYG{o}{.}\PYG{n}{IntegerPartition}\PYG{p}{(}\PYG{p}{[}\PYG{l+m+mi}{3}\PYG{p}{,}\PYG{l+m+mi}{2}\PYG{p}{]}\PYG{p}{)}\PYG{p}{,}\PYG{l+m+mi}{1}\PYG{p}{)}
\PYG{g+gp}{\PYGZgt{}\PYGZgt{}\PYGZgt{} }\PYG{n}{tau}\PYG{o}{.}\PYG{n}{set}\PYG{p}{(}\PYG{n}{Snob2}\PYG{o}{.}\PYG{n}{IntegerPartition}\PYG{p}{(}\PYG{p}{[}\PYG{l+m+mi}{3}\PYG{p}{,}\PYG{l+m+mi}{1}\PYG{p}{,}\PYG{l+m+mi}{1}\PYG{p}{]}\PYG{p}{)}\PYG{p}{,}\PYG{l+m+mi}{1}\PYG{p}{)}
\PYG{g+gp}{\PYGZgt{}\PYGZgt{}\PYGZgt{} }\PYG{k}{print}\PYG{p}{(}\PYG{n}{tau}\PYG{p}{)}
\PYG{g+go}{([4,1]:2,[3,2]:1,[3,1,1]:1)}
\end{sphinxVerbatim}


\section{Sn\sphinxhyphen{}functions and Sn\sphinxhyphen{}vectors}
\label{\detokenize{index:sn-functions-and-sn-vectors}}

\subsection{Sn functions}
\label{\detokenize{index:sn-functions}}
\sphinxAtStartPar
The class \sphinxcode{\sphinxupquote{SnFunction}} represents functions on \(\mathbb{S}_n\).
The following initializes a function on \(\mathbb{S}_3\) with random Gaussian entries and
prints it out.

\begin{sphinxVerbatim}[commandchars=\\\{\}]
\PYG{g+gp}{\PYGZgt{}\PYGZgt{}\PYGZgt{} }\PYG{n}{f}\PYG{o}{=}\PYG{n}{Snob2}\PYG{o}{.}\PYG{n}{SnFunction}\PYG{o}{.}\PYG{n}{gaussian}\PYG{p}{(}\PYG{l+m+mi}{3}\PYG{p}{)}
\PYG{g+gp}{\PYGZgt{}\PYGZgt{}\PYGZgt{} }\PYG{k}{print}\PYG{p}{(}\PYG{n}{f}\PYG{p}{)}
\PYG{g+go}{[ 1 2 3 ] : \PYGZhy{}1.23974}
\PYG{g+go}{[ 2 1 3 ] : \PYGZhy{}0.407472}
\PYG{g+go}{[ 1 3 2 ] : 1.61201}
\PYG{g+go}{[ 2 3 1 ] : 0.399771}
\PYG{g+go}{[ 3 1 2 ] : 1.3828}
\PYG{g+go}{[ 3 2 1 ] : 0.0523187}
\end{sphinxVerbatim}

\sphinxAtStartPar
The value of \sphinxcode{\sphinxupquote{f}} at specific group elements can be accessed via the \sphinxcode{\sphinxupquote{SnElement}} object or just its index.

\begin{sphinxVerbatim}[commandchars=\\\{\}]
\PYG{g+gp}{\PYGZgt{}\PYGZgt{}\PYGZgt{} }\PYG{n}{f}\PYG{p}{[}\PYG{n}{Snob2}\PYG{o}{.}\PYG{n}{SnElement}\PYG{p}{(}\PYG{p}{[}\PYG{l+m+mi}{1}\PYG{p}{,}\PYG{l+m+mi}{3}\PYG{p}{,}\PYG{l+m+mi}{2}\PYG{p}{]}\PYG{p}{)}\PYG{p}{]}
\PYG{g+go}{1.6120094060897827}
\PYG{g+go}{0.1949467808008194}
\PYG{g+gp}{\PYGZgt{}\PYGZgt{}\PYGZgt{} }\PYG{n}{f}\PYG{p}{[}\PYG{l+m+mi}{2}\PYG{p}{]}
\PYG{g+go}{1.6120094060897827}
\end{sphinxVerbatim}

\sphinxAtStartPar
The \sphinxstyleemphasis{left\sphinxhyphen{}translate} of \(f\) by a permutation \(\pi\) is defined \(g_1(\sigma)=f(\pi^{-1}\sigma)\).

\begin{sphinxVerbatim}[commandchars=\\\{\}]
\PYG{g+gp}{\PYGZgt{}\PYGZgt{}\PYGZgt{} }\PYG{n}{g1}\PYG{o}{=}\PYG{n}{f}\PYG{o}{.}\PYG{n}{left\PYGZus{}translate}\PYG{p}{(}\PYG{n}{pi}\PYG{p}{)}
\PYG{g+go}{SnFunction moved}
\PYG{g+gp}{\PYGZgt{}\PYGZgt{}\PYGZgt{} }\PYG{k}{print}\PYG{p}{(}\PYG{n}{g1}\PYG{p}{)}
\PYG{g+go}{[ 1 2 3 ] : \PYGZhy{}0.407472}
\PYG{g+go}{[ 2 1 3 ] : \PYGZhy{}1.23974}
\PYG{g+go}{[ 1 3 2 ] : 0.399771}
\PYG{g+go}{[ 2 3 1 ] : 1.61201}
\PYG{g+go}{[ 3 1 2 ] : 0.0523187}
\PYG{g+go}{[ 3 2 1 ] : 1.3828}
\end{sphinxVerbatim}

\sphinxAtStartPar
The \sphinxstyleemphasis{right\sphinxhyphen{}translate} of \(f\) by a permutation \(\pi\) is defined \(g_1(\sigma)=f(\sigma \pi^{-1})\).

\begin{sphinxVerbatim}[commandchars=\\\{\}]
\PYG{g+gp}{\PYGZgt{}\PYGZgt{}\PYGZgt{} }\PYG{n}{g2}\PYG{o}{=}\PYG{n}{f}\PYG{o}{.}\PYG{n}{right\PYGZus{}translate}\PYG{p}{(}\PYG{n}{pi}\PYG{p}{)}
\PYG{g+go}{SnFunction moved}
\PYG{g+gp}{\PYGZgt{}\PYGZgt{}\PYGZgt{} }\PYG{k}{print}\PYG{p}{(}\PYG{n}{g2}\PYG{p}{)}
\PYG{g+go}{[ 1 2 3 ] : \PYGZhy{}0.407472}
\PYG{g+go}{[ 2 1 3 ] : \PYGZhy{}1.23974}
\PYG{g+go}{[ 1 3 2 ] : 1.3828}
\PYG{g+go}{[ 2 3 1 ] : 0.0523187}
\PYG{g+go}{[ 3 1 2 ] : 1.61201}
\PYG{g+go}{[ 3 2 1 ] : 0.399771}
\end{sphinxVerbatim}

\sphinxAtStartPar
The \sphinxstyleemphasis{inverse} of \(f\) is defined \(f^{-1}(\sigma)=f(\sigma)\).

\begin{sphinxVerbatim}[commandchars=\\\{\}]
\PYG{g+gp}{\PYGZgt{}\PYGZgt{}\PYGZgt{} }\PYG{n}{f}\PYG{o}{=}\PYG{n}{Snob2}\PYG{o}{.}\PYG{n}{SnFunction}\PYG{o}{.}\PYG{n}{gaussian}\PYG{p}{(}\PYG{l+m+mi}{3}\PYG{p}{)}
\PYG{g+gp}{\PYGZgt{}\PYGZgt{}\PYGZgt{} }\PYG{n}{finv}\PYG{o}{=}\PYG{n}{f}\PYG{o}{.}\PYG{n}{inv}\PYG{p}{(}\PYG{p}{)}
\PYG{g+gp}{\PYGZgt{}\PYGZgt{}\PYGZgt{} }\PYG{k}{print}\PYG{p}{(}\PYG{n}{finv}\PYG{p}{)}
\PYG{g+go}{[ 1 2 3 ] : \PYGZhy{}1.23974}
\PYG{g+go}{[ 2 1 3 ] : \PYGZhy{}0.407472}
\PYG{g+go}{[ 1 3 2 ] : 1.61201}
\PYG{g+go}{[ 2 3 1 ] : 1.3828}
\PYG{g+go}{[ 3 1 2 ] : 0.399771}
\PYG{g+go}{[ 3 2 1 ] : 0.0523187}
\end{sphinxVerbatim}


\subsection{Sn/Sm functions}
\label{\detokenize{index:sn-sm-functions}}
\sphinxAtStartPar
The class \sphinxcode{\sphinxupquote{SnOverSmFunction}} represents functions on \(\mathbb{S}_n/\mathbb{S}_m\).
The following initializes a function on \(\mathbb{S}_5/\mathbb{S}_4\) with random Gaussian entries and
prints it out.

\begin{sphinxVerbatim}[commandchars=\\\{\}]
\PYG{g+gp}{\PYGZgt{}\PYGZgt{}\PYGZgt{} }\PYG{n}{f}\PYG{o}{=}\PYG{n}{Snob2}\PYG{o}{.}\PYG{n}{SnOverSmFunction}\PYG{o}{.}\PYG{n}{gaussian}\PYG{p}{(}\PYG{l+m+mi}{5}\PYG{p}{,}\PYG{l+m+mi}{4}\PYG{p}{)}
\PYG{g+gp}{\PYGZgt{}\PYGZgt{}\PYGZgt{} }\PYG{k}{print}\PYG{p}{(}\PYG{n}{f}\PYG{p}{)}
\PYG{g+go}{0.74589}
\PYG{g+go}{\PYGZhy{}1.75177}
\PYG{g+go}{\PYGZhy{}0.965146}
\PYG{g+go}{\PYGZhy{}0.474282}
\PYG{g+go}{\PYGZhy{}0.546571}
\end{sphinxVerbatim}


\subsection{Sn class functions}
\label{\detokenize{index:sn-class-functions}}
\sphinxAtStartPar
The class \sphinxcode{\sphinxupquote{SnClassFunction}} represents functions on the conjugacy classes of \(\mathbb{S}_n\).
An important example of class functions are the characters of the group.
The following initializes a class function on \(\mathbb{S}_4\) with random Gaussian entries
and prints it out.

\begin{sphinxVerbatim}[commandchars=\\\{\}]
\PYG{g+gp}{\PYGZgt{}\PYGZgt{}\PYGZgt{} }\PYG{n}{f}\PYG{o}{=}\PYG{n}{Snob2}\PYG{o}{.}\PYG{n}{SnClassFunction}\PYG{o}{.}\PYG{n}{gaussian}\PYG{p}{(}\PYG{l+m+mi}{4}\PYG{p}{)}
\PYG{g+gp}{\PYGZgt{}\PYGZgt{}\PYGZgt{} }\PYG{k}{print}\PYG{p}{(}\PYG{n}{f}\PYG{p}{)}
\PYG{g+go}{SnCClass[4] : \PYGZhy{}1.23974}
\PYG{g+go}{SnCClass[3,1] : \PYGZhy{}0.407472}
\PYG{g+go}{SnCClass[2,2] : 1.61201}
\PYG{g+go}{SnCClass[2,1,1] : 0.399771}
\PYG{g+go}{SnCClass[1,1,1,1] : 1.3828}
\end{sphinxVerbatim}

\sphinxAtStartPar
The value of \sphinxcode{\sphinxupquote{f}} at specific conjugacy classes can be accessed via the corresponding \sphinxcode{\sphinxupquote{SnCClass}},
\sphinxcode{\sphinxupquote{IntegerPartition}} or just the index.

\begin{sphinxVerbatim}[commandchars=\\\{\}]
\PYG{g+gp}{\PYGZgt{}\PYGZgt{}\PYGZgt{} }\PYG{n}{f}\PYG{p}{[}\PYG{n}{Snob2}\PYG{o}{.}\PYG{n}{SnCClass}\PYG{p}{(}\PYG{p}{[}\PYG{l+m+mi}{2}\PYG{p}{,}\PYG{l+m+mi}{2}\PYG{p}{]}\PYG{p}{)}\PYG{p}{]}
\PYG{g+go}{1.6120094060897827}
\PYG{g+gp}{\PYGZgt{}\PYGZgt{}\PYGZgt{} }\PYG{n}{f}\PYG{p}{[}\PYG{n}{Snob2}\PYG{o}{.}\PYG{n}{IntegerPartition}\PYG{p}{(}\PYG{p}{[}\PYG{l+m+mi}{2}\PYG{p}{,}\PYG{l+m+mi}{2}\PYG{p}{]}\PYG{p}{)}\PYG{p}{]}
\PYG{g+go}{1.6120094060897827}
\PYG{g+gp}{\PYGZgt{}\PYGZgt{}\PYGZgt{} }\PYG{n}{f}\PYG{p}{[}\PYG{n}{Snob2}\PYG{o}{.}\PYG{n}{SnCClass}\PYG{p}{(}\PYG{l+m+mi}{2}\PYG{p}{)}\PYG{p}{]}
\PYG{g+go}{1.6120094060897827}
\end{sphinxVerbatim}


\subsection{Sn parts}
\label{\detokenize{index:sn-parts}}
\sphinxAtStartPar
An \sphinxcode{\sphinxupquote{SnPart}} of type \(\lambda\) is a collection of \(m\) vectors on which acts
by the irreducible representation \(\rho_\lambda\). The \sphinxcode{\sphinxupquote{SnPart}} is stored as a matrix
\(\mathbb{R}^{d_\lambda\times m}\).

\begin{sphinxVerbatim}[commandchars=\\\{\}]
\PYG{g+go}{\PYGZgt{}\PYGZgt{}\PYGZgt{}lambd=Snob2.IntegerPartition([3,2])}
\PYG{g+gp}{\PYGZgt{}\PYGZgt{}\PYGZgt{} }\PYG{n}{p}\PYG{o}{=}\PYG{n}{Snob2}\PYG{o}{.}\PYG{n}{SnPart}\PYG{o}{.}\PYG{n}{gaussian}\PYG{p}{(}\PYG{n}{lambd}\PYG{p}{,}\PYG{l+m+mi}{3}\PYG{p}{)}
\PYG{g+gp}{\PYGZgt{}\PYGZgt{}\PYGZgt{} }\PYG{k}{print}\PYG{p}{(}\PYG{n}{p}\PYG{p}{)}
\PYG{g+go}{Part [3,2]:}
\PYG{g+go}{[ \PYGZhy{}1.23974 \PYGZhy{}0.407472 1.61201 ]}
\PYG{g+go}{[ 0.399771 1.3828 0.0523187 ]}
\PYG{g+go}{[ \PYGZhy{}0.904146 1.87065 \PYGZhy{}1.66043 ]}
\PYG{g+go}{[ \PYGZhy{}0.688081 0.0757219 1.47339 ]}
\PYG{g+go}{[ 0.097221 \PYGZhy{}0.89237 \PYGZhy{}0.228782 ]}
\end{sphinxVerbatim}


\subsection{Sn vectors}
\label{\detokenize{index:sn-vectors}}
\sphinxAtStartPar
An Sn covariant vector or \sphinxstyleemphasis{Sn\sphinxhyphen{}vector} for short is a vector that transforms under the action of
\(\mathbb{S}_n\) by a combination of irreducible representations.
Sn\sphinxhyphen{}vectors are stored as \sphinxcode{\sphinxupquote{SnVec}} objects as a list of {\color{red}\bfseries{}\textasciigrave{}\textasciigrave{}}SnPart\textasciigrave{}\textasciigrave{}s.

\begin{sphinxVerbatim}[commandchars=\\\{\}]
\PYG{g+gp}{\PYGZgt{}\PYGZgt{}\PYGZgt{} }\PYG{n}{tau}\PYG{o}{=}\PYG{n}{Snob2}\PYG{o}{.}\PYG{n}{SnType}\PYG{p}{(}\PYG{n}{Snob2}\PYG{o}{.}\PYG{n}{IntegerPartition}\PYG{p}{(}\PYG{p}{[}\PYG{l+m+mi}{4}\PYG{p}{,}\PYG{l+m+mi}{1}\PYG{p}{]}\PYG{p}{)}\PYG{p}{,}\PYG{l+m+mi}{2}\PYG{p}{)}
\PYG{g+gp}{\PYGZgt{}\PYGZgt{}\PYGZgt{} }\PYG{n}{tau}\PYG{o}{.}\PYG{n}{set}\PYG{p}{(}\PYG{n}{Snob2}\PYG{o}{.}\PYG{n}{IntegerPartition}\PYG{p}{(}\PYG{p}{[}\PYG{l+m+mi}{3}\PYG{p}{,}\PYG{l+m+mi}{2}\PYG{p}{]}\PYG{p}{)}\PYG{p}{,}\PYG{l+m+mi}{1}\PYG{p}{)}
\PYG{g+gp}{\PYGZgt{}\PYGZgt{}\PYGZgt{} }\PYG{n}{tau}\PYG{o}{.}\PYG{n}{set}\PYG{p}{(}\PYG{n}{Snob2}\PYG{o}{.}\PYG{n}{IntegerPartition}\PYG{p}{(}\PYG{p}{[}\PYG{l+m+mi}{3}\PYG{p}{,}\PYG{l+m+mi}{1}\PYG{p}{,}\PYG{l+m+mi}{1}\PYG{p}{]}\PYG{p}{)}\PYG{p}{,}\PYG{l+m+mi}{1}\PYG{p}{)}
\PYG{g+gp}{\PYGZgt{}\PYGZgt{}\PYGZgt{} }\PYG{n}{v}\PYG{o}{=}\PYG{n}{Snob2}\PYG{o}{.}\PYG{n}{SnVec}\PYG{o}{.}\PYG{n}{gaussian}\PYG{p}{(}\PYG{n}{tau}\PYG{p}{)}
\PYG{g+gp}{\PYGZgt{}\PYGZgt{}\PYGZgt{} }\PYG{k}{print}\PYG{p}{(}\PYG{n}{v}\PYG{p}{)}
\PYG{g+go}{Part [4,1]:}
\PYG{g+go}{[ \PYGZhy{}1.23974 \PYGZhy{}0.407472 ]}
\PYG{g+go}{[ 1.61201 0.399771 ]}
\PYG{g+go}{[ 1.3828 0.0523187 ]}
\PYG{g+go}{[ \PYGZhy{}0.904146 1.87065 ]}

\PYG{g+go}{Part [3,2]:}
\PYG{g+go}{[ \PYGZhy{}1.66043 ]}
\PYG{g+go}{[ \PYGZhy{}0.688081 ]}
\PYG{g+go}{[ 0.0757219 ]}
\PYG{g+go}{[ 1.47339 ]}
\PYG{g+go}{[ 0.097221 ]}

\PYG{g+go}{Part [3,1,1]:}
\PYG{g+go}{[ \PYGZhy{}0.228782 ]}
\PYG{g+go}{[ 1.16493 ]}
\PYG{g+go}{[ 0.584898 ]}
\PYG{g+go}{[ \PYGZhy{}0.660558 ]}
\PYG{g+go}{[ 0.534755 ]}
\PYG{g+go}{[ \PYGZhy{}0.607787 ]}
\end{sphinxVerbatim}


\section{Sn Fourier transforms}
\label{\detokenize{index:sn-fourier-transforms}}
\sphinxAtStartPar
The Fourier transform on \(\mathbb{S}_n\) converts a function on \(\mathbb{S}_n\)
or a qutient space of \(\mathbb{S}_n\) into an \(\mathbb{S}_n\)\textendash{}vector.
Snob2 uses Clausen’s FFT to compute forward and backward Fourier transforms.
Fourier transforms employs several internal data structures that can be reused on future transforms.
Therefore before conducting a Fourier transform a corresponding \sphinxcode{\sphinxupquote{ClausenFFT}} must be constructed.


\subsection{FFTs on Sn}
\label{\detokenize{index:ffts-on-sn}}
\sphinxAtStartPar
The following sets up an \sphinxcode{\sphinxupquote{ClausenFFT}} object for Fourier transformation on \(\mathbb{S}_4\) and
defines a random function on the group.

\begin{sphinxVerbatim}[commandchars=\\\{\}]
\PYG{g+gp}{\PYGZgt{}\PYGZgt{}\PYGZgt{} }\PYG{n}{fft}\PYG{o}{=}\PYG{n}{Snob2}\PYG{o}{.}\PYG{n}{ClausenFFT}\PYG{p}{(}\PYG{l+m+mi}{4}\PYG{p}{)}
\PYG{g+gp}{\PYGZgt{}\PYGZgt{}\PYGZgt{} }\PYG{n}{f}\PYG{o}{=}\PYG{n}{Snob2}\PYG{o}{.}\PYG{n}{SnFunction}\PYG{o}{.}\PYG{n}{gaussian}\PYG{p}{(}\PYG{l+m+mi}{4}\PYG{p}{)}
\PYG{g+gp}{\PYGZgt{}\PYGZgt{}\PYGZgt{} }\PYG{k}{print}\PYG{p}{(}\PYG{n}{f}\PYG{p}{)}
\PYG{g+go}{[ 1 2 3 4 ] : \PYGZhy{}1.23974}
\PYG{g+go}{[ 2 1 3 4 ] : \PYGZhy{}0.407472}
\PYG{g+go}{[ 1 3 2 4 ] : 1.61201}
\PYG{g+go}{[ 2 3 1 4 ] : 0.399771}
\PYG{g+go}{[ 3 1 2 4 ] : 1.3828}
\PYG{g+go}{[ 3 2 1 4 ] : 0.0523187}
\PYG{g+go}{[ 1 2 4 3 ] : \PYGZhy{}0.904146}
\PYG{g+go}{[ 2 1 4 3 ] : 1.87065}
\PYG{g+go}{[ 1 3 4 2 ] : \PYGZhy{}1.66043}
\PYG{g+go}{[ 2 3 4 1 ] : \PYGZhy{}0.688081}
\PYG{g+go}{[ 3 1 4 2 ] : 0.0757219}
\PYG{g+go}{[ 3 2 4 1 ] : 1.47339}
\PYG{g+go}{[ 1 4 2 3 ] : 0.097221}
\PYG{g+go}{[ 2 4 1 3 ] : \PYGZhy{}0.89237}
\PYG{g+go}{[ 1 4 3 2 ] : \PYGZhy{}0.228782}
\PYG{g+go}{[ 2 4 3 1 ] : 1.16493}
\PYG{g+go}{[ 3 4 1 2 ] : 0.584898}
\PYG{g+go}{[ 3 4 2 1 ] : \PYGZhy{}0.660558}
\PYG{g+go}{[ 4 1 2 3 ] : 0.534755}
\PYG{g+go}{[ 4 2 1 3 ] : \PYGZhy{}0.607787}
\PYG{g+go}{[ 4 1 3 2 ] : 0.74589}
\PYG{g+go}{[ 4 2 3 1 ] : \PYGZhy{}1.75177}
\PYG{g+go}{[ 4 3 1 2 ] : \PYGZhy{}0.965146}
\PYG{g+go}{[ 4 3 2 1 ] : \PYGZhy{}0.474282}
\end{sphinxVerbatim}

\sphinxAtStartPar
We can now use our fft object to take the Fourier transform of f.

\begin{sphinxVerbatim}[commandchars=\\\{\}]
\PYG{g+gp}{\PYGZgt{}\PYGZgt{}\PYGZgt{} }\PYG{n}{F}\PYG{o}{=}\PYG{n}{fft}\PYG{p}{(}\PYG{n}{f}\PYG{p}{)}
\PYG{g+gp}{\PYGZgt{}\PYGZgt{}\PYGZgt{} }\PYG{k}{print}\PYG{p}{(}\PYG{n}{F}\PYG{p}{)}
\PYG{g+go}{Part [4]:}
\PYG{g+go}{[ \PYGZhy{}0.486197 ]}

\PYG{g+go}{Part [3,1]:}
\PYG{g+go}{[ 2.56166 1.21663 \PYGZhy{}0.41762 ]}
\PYG{g+go}{[ 1.3139 \PYGZhy{}1.81861 3.2474 ]}
\PYG{g+go}{[ 2.10957 \PYGZhy{}3.31125 \PYGZhy{}5.47569 ]}

\PYG{g+go}{Part [2,2]:}
\PYG{g+go}{[ \PYGZhy{}3.05059 \PYGZhy{}2.65296 ]}
\PYG{g+go}{[ 1.56762 1.53786 ]}

\PYG{g+go}{Part [2,1,1]:}
\PYG{g+go}{[ \PYGZhy{}5.13609 4.39341 1.45563 ]}
\PYG{g+go}{[ 3.59791 2.07342 \PYGZhy{}0.436283 ]}
\PYG{g+go}{[ \PYGZhy{}3.65454 0.381513 \PYGZhy{}3.60564 ]}

\PYG{g+go}{Part [1,1,1,1]:}
\PYG{g+go}{[ 7.96084 ]}
\end{sphinxVerbatim}

\sphinxAtStartPar
The inverse Fourier transform can be computed with the same FFT object and should return the
original function \sphinxcode{\sphinxupquote{f}}.

\begin{sphinxVerbatim}[commandchars=\\\{\}]
\PYG{g+gp}{\PYGZgt{}\PYGZgt{}\PYGZgt{} }\PYG{n}{fd}\PYG{o}{=}\PYG{n}{fft}\PYG{o}{.}\PYG{n}{inv}\PYG{p}{(}\PYG{n}{F}\PYG{p}{)}
\PYG{g+gp}{\PYGZgt{}\PYGZgt{}\PYGZgt{} }\PYG{k}{print}\PYG{p}{(}\PYG{n}{fd}\PYG{p}{)}
\PYG{g+go}{[ 1 2 3 4 ] : \PYGZhy{}1.23974}
\PYG{g+go}{[ 2 1 3 4 ] : \PYGZhy{}0.407472}
\PYG{g+go}{[ 1 3 2 4 ] : 1.61201}
\PYG{g+go}{[ 2 3 1 4 ] : 0.399771}
\PYG{g+go}{[ 3 1 2 4 ] : 1.3828}
\PYG{g+go}{[ 3 2 1 4 ] : 0.0523185}
\PYG{g+go}{[ 1 2 4 3 ] : \PYGZhy{}0.904147}
\PYG{g+go}{[ 2 1 4 3 ] : 1.87065}
\PYG{g+go}{[ 1 3 4 2 ] : \PYGZhy{}1.66043}
\PYG{g+go}{[ 2 3 4 1 ] : \PYGZhy{}0.688081}
\PYG{g+go}{[ 3 1 4 2 ] : 0.0757219}
\PYG{g+go}{[ 3 2 4 1 ] : 1.47339}
\PYG{g+go}{[ 1 4 2 3 ] : 0.097221}
\PYG{g+go}{[ 2 4 1 3 ] : \PYGZhy{}0.89237}
\PYG{g+go}{[ 1 4 3 2 ] : \PYGZhy{}0.228782}
\PYG{g+go}{[ 2 4 3 1 ] : 1.16493}
\PYG{g+go}{[ 3 4 1 2 ] : 0.584898}
\PYG{g+go}{[ 3 4 2 1 ] : \PYGZhy{}0.660558}
\PYG{g+go}{[ 4 1 2 3 ] : 0.534755}
\PYG{g+go}{[ 4 2 1 3 ] : \PYGZhy{}0.607787}
\PYG{g+go}{[ 4 1 3 2 ] : 0.74589}
\PYG{g+go}{[ 4 2 3 1 ] : \PYGZhy{}1.75177}
\PYG{g+go}{[ 4 3 1 2 ] : \PYGZhy{}0.965146}
\PYG{g+go}{[ 4 3 2 1 ] : \PYGZhy{}0.474282}
\end{sphinxVerbatim}


\subsection{FFTs on Sn/Sm}
\label{\detokenize{index:ffts-on-sn-sm}}
\sphinxAtStartPar
The \sphinxcode{\sphinxupquote{ClausenFFT}} can also be used to compute FFTs on \(\mathbb{S}_n/\mathbb{S}_m\).

\begin{sphinxVerbatim}[commandchars=\\\{\}]
\PYG{g+gp}{\PYGZgt{}\PYGZgt{}\PYGZgt{} }\PYG{n}{fft}\PYG{o}{=}\PYG{n}{Snob2}\PYG{o}{.}\PYG{n}{ClausenFFT}\PYG{p}{(}\PYG{l+m+mi}{4}\PYG{p}{,}\PYG{l+m+mi}{2}\PYG{p}{)}
\PYG{g+gp}{\PYGZgt{}\PYGZgt{}\PYGZgt{} }\PYG{n}{f}\PYG{o}{=}\PYG{n}{Snob2}\PYG{o}{.}\PYG{n}{SnOverSmFunction}\PYG{o}{.}\PYG{n}{gaussian}\PYG{p}{(}\PYG{l+m+mi}{4}\PYG{p}{,}\PYG{l+m+mi}{2}\PYG{p}{)}
\PYG{g+gp}{\PYGZgt{}\PYGZgt{}\PYGZgt{} }\PYG{k}{print}\PYG{p}{(}\PYG{n}{f}\PYG{p}{)}
\PYG{g+go}{\PYGZhy{}0.546571}
\PYG{g+go}{\PYGZhy{}0.0384917}
\PYG{g+go}{0.194947}
\PYG{g+go}{\PYGZhy{}0.485144}
\PYG{g+go}{\PYGZhy{}0.370271}
\PYG{g+go}{\PYGZhy{}1.12408}
\PYG{g+go}{1.73664}
\PYG{g+go}{0.882195}
\PYG{g+go}{\PYGZhy{}1.50279}
\PYG{g+go}{0.570759}
\PYG{g+go}{\PYGZhy{}0.929941}
\PYG{g+go}{\PYGZhy{}0.934988}

\PYG{g+gp}{\PYGZgt{}\PYGZgt{}\PYGZgt{} }\PYG{n}{F}\PYG{o}{=}\PYG{n}{fft}\PYG{p}{(}\PYG{n}{f}\PYG{p}{)}
\PYG{g+gp}{\PYGZgt{}\PYGZgt{}\PYGZgt{} }\PYG{k}{print}\PYG{p}{(}\PYG{n}{F}\PYG{p}{)}
\PYG{g+go}{Part [4]:}
\PYG{g+go}{[ \PYGZhy{}2.54774 ]}

\PYG{g+go}{Part [3,1]:}
\PYG{g+go}{[ 0.329091 3.59416 ]}
\PYG{g+go}{[ \PYGZhy{}1.78231 0.663375 ]}
\PYG{g+go}{[ 1.96793 1.63815 ]}

\PYG{g+go}{Part [2,2]:}
\PYG{g+go}{[ \PYGZhy{}3.93037 ]}
\PYG{g+go}{[ \PYGZhy{}1.41466 ]}

\PYG{g+go}{Part [2,1,1]:}
\PYG{g+go}{[ 0.290743 ]}
\PYG{g+go}{[ \PYGZhy{}1.23415 ]}
\PYG{g+go}{[ \PYGZhy{}1.32773 ]}
\end{sphinxVerbatim}

\begin{sphinxVerbatim}[commandchars=\\\{\}]
\PYG{g+gp}{\PYGZgt{}\PYGZgt{}\PYGZgt{} }\PYG{n}{fd}\PYG{o}{=}\PYG{n}{fft}\PYG{o}{.}\PYG{n}{inv\PYGZus{}snsm}\PYG{p}{(}\PYG{n}{F}\PYG{p}{)}
\PYG{g+go}{SnFunction moved}
\PYG{g+gp}{\PYGZgt{}\PYGZgt{}\PYGZgt{} }\PYG{k}{print}\PYG{p}{(}\PYG{n}{fd}\PYG{p}{)}
\PYG{g+go}{[ 1 2 3 ] : \PYGZhy{}0.546571}
\PYG{g+go}{[ 2 1 3 ] : \PYGZhy{}0.0384917}
\PYG{g+go}{[ 1 3 2 ] : 0.194947}
\PYG{g+go}{[ 2 3 1 ] : \PYGZhy{}0.485144}
\PYG{g+go}{[ 3 1 2 ] : \PYGZhy{}0.37027}
\PYG{g+go}{[ 3 2 1 ] : \PYGZhy{}1.12408}
\PYG{g+go}{[ 32705 1 2 ] : 1.73664}
\PYG{g+go}{[ 3 2 1 ] : 0.882195}
\PYG{g+go}{[ 3 1 2 ] : \PYGZhy{}1.50279}
\PYG{g+go}{[ 519242688 2 1 ] : 0.570759}
\PYG{g+go}{[ 3 1 2 ] : \PYGZhy{}0.929941}
\PYG{g+go}{[ 3 2 1 ] : \PYGZhy{}0.934988}
\end{sphinxVerbatim}

\begin{DUlineblock}{0em}
\item[] 
\end{DUlineblock}


\chapter{Reference}
\label{\detokenize{index:reference}}

\section{Combinatorial classes}
\label{\detokenize{index:id3}}
\begin{DUlineblock}{0em}
\item[] 
\end{DUlineblock}
\index{IntegerPartition (class in Snob2)@\spxentry{IntegerPartition}\spxextra{class in Snob2}}

\begin{fulllineitems}
\phantomsection\label{\detokenize{index:Snob2.IntegerPartition}}\pysiglinewithargsret{\sphinxbfcode{\sphinxupquote{class }}\sphinxbfcode{\sphinxupquote{IntegerPartition}}}{\emph{\DUrole{n}{parts}}}{}
\sphinxAtStartPar
Class to represent an integer partition (p\_1,p\_2,…,p\_k) of n.
\index{\_\_getitem\_\_() (IntegerPartition method)@\spxentry{\_\_getitem\_\_()}\spxextra{IntegerPartition method}}

\begin{fulllineitems}
\phantomsection\label{\detokenize{index:Snob2.IntegerPartition.__getitem__}}\pysiglinewithargsret{\sphinxbfcode{\sphinxupquote{\_\_getitem\_\_}}}{\emph{\DUrole{n}{self}\DUrole{p}{:} \DUrole{n}{{\hyperref[\detokenize{index:Snob2.IntegerPartition}]{\sphinxcrossref{Snob2.IntegerPartition}}}}}, \emph{\DUrole{n}{arg0}\DUrole{p}{:} \DUrole{n}{\sphinxhref{https://docs.python.org/3/library/functions.html\#int}{int}}}}{{ $\rightarrow$ \sphinxhref{https://docs.python.org/3/library/functions.html\#int}{int}}}
\sphinxAtStartPar
Return the the i’th part, p\_i.

\end{fulllineitems}

\index{\_\_init\_\_() (IntegerPartition method)@\spxentry{\_\_init\_\_()}\spxextra{IntegerPartition method}}

\begin{fulllineitems}
\phantomsection\label{\detokenize{index:Snob2.IntegerPartition.__init__}}\pysiglinewithargsret{\sphinxbfcode{\sphinxupquote{\_\_init\_\_}}}{\emph{\DUrole{n}{self}\DUrole{p}{:} \DUrole{n}{{\hyperref[\detokenize{index:Snob2.IntegerPartition}]{\sphinxcrossref{Snob2.IntegerPartition}}}}}, \emph{\DUrole{n}{arg0}\DUrole{p}{:} \DUrole{n}{List\DUrole{p}{{[}}\sphinxhref{https://docs.python.org/3/library/functions.html\#int}{int}\DUrole{p}{{]}}}}}{{ $\rightarrow$ \sphinxhref{https://docs.python.org/3/library/constants.html\#None}{None}}}
\sphinxAtStartPar
Initialize from a list of integers.

\end{fulllineitems}

\index{\_\_setitem\_\_() (IntegerPartition method)@\spxentry{\_\_setitem\_\_()}\spxextra{IntegerPartition method}}

\begin{fulllineitems}
\phantomsection\label{\detokenize{index:Snob2.IntegerPartition.__setitem__}}\pysiglinewithargsret{\sphinxbfcode{\sphinxupquote{\_\_setitem\_\_}}}{\emph{\DUrole{n}{self}\DUrole{p}{:} \DUrole{n}{{\hyperref[\detokenize{index:Snob2.IntegerPartition}]{\sphinxcrossref{Snob2.IntegerPartition}}}}}, \emph{\DUrole{n}{arg0}\DUrole{p}{:} \DUrole{n}{\sphinxhref{https://docs.python.org/3/library/functions.html\#int}{int}}}, \emph{\DUrole{n}{arg1}\DUrole{p}{:} \DUrole{n}{\sphinxhref{https://docs.python.org/3/library/functions.html\#int}{int}}}}{{ $\rightarrow$ \sphinxhref{https://docs.python.org/3/library/constants.html\#None}{None}}}
\sphinxAtStartPar
Set the i’th part to x

\end{fulllineitems}

\index{\_\_str\_\_() (IntegerPartition method)@\spxentry{\_\_str\_\_()}\spxextra{IntegerPartition method}}

\begin{fulllineitems}
\phantomsection\label{\detokenize{index:Snob2.IntegerPartition.__str__}}\pysiglinewithargsret{\sphinxbfcode{\sphinxupquote{\_\_str\_\_}}}{\emph{\DUrole{n}{self}\DUrole{p}{:} \DUrole{n}{{\hyperref[\detokenize{index:Snob2.IntegerPartition}]{\sphinxcrossref{Snob2.IntegerPartition}}}}}, \emph{\DUrole{n}{indent}\DUrole{p}{:} \DUrole{n}{\sphinxhref{https://docs.python.org/3/library/stdtypes.html\#str}{str}} \DUrole{o}{=} \DUrole{default_value}{\textquotesingle{}\textquotesingle{}}}}{{ $\rightarrow$ \sphinxhref{https://docs.python.org/3/library/stdtypes.html\#str}{str}}}
\sphinxAtStartPar
Print the integer partition to string.

\end{fulllineitems}

\index{getn() (IntegerPartition method)@\spxentry{getn()}\spxextra{IntegerPartition method}}

\begin{fulllineitems}
\phantomsection\label{\detokenize{index:Snob2.IntegerPartition.getn}}\pysiglinewithargsret{\sphinxbfcode{\sphinxupquote{getn}}}{\emph{\DUrole{n}{self}\DUrole{p}{:} \DUrole{n}{{\hyperref[\detokenize{index:Snob2.IntegerPartition}]{\sphinxcrossref{Snob2.IntegerPartition}}}}}}{{ $\rightarrow$ \sphinxhref{https://docs.python.org/3/library/functions.html\#int}{int}}}
\sphinxAtStartPar
Return n.

\end{fulllineitems}

\index{height() (IntegerPartition method)@\spxentry{height()}\spxextra{IntegerPartition method}}

\begin{fulllineitems}
\phantomsection\label{\detokenize{index:Snob2.IntegerPartition.height}}\pysiglinewithargsret{\sphinxbfcode{\sphinxupquote{height}}}{\emph{\DUrole{n}{self}\DUrole{p}{:} \DUrole{n}{{\hyperref[\detokenize{index:Snob2.IntegerPartition}]{\sphinxcrossref{Snob2.IntegerPartition}}}}}}{{ $\rightarrow$ \sphinxhref{https://docs.python.org/3/library/functions.html\#int}{int}}}
\sphinxAtStartPar
Return the number of parts, k.

\end{fulllineitems}


\end{fulllineitems}


\begin{DUlineblock}{0em}
\item[] 
\end{DUlineblock}
\index{IntegerPartitions (class in Snob2)@\spxentry{IntegerPartitions}\spxextra{class in Snob2}}

\begin{fulllineitems}
\phantomsection\label{\detokenize{index:Snob2.IntegerPartitions}}\pysiglinewithargsret{\sphinxbfcode{\sphinxupquote{class }}\sphinxbfcode{\sphinxupquote{IntegerPartitions}}}{\emph{\DUrole{n}{n}}}{}
\sphinxAtStartPar
This object represents all integer partitions of a given integer n.
\index{\_\_getitem\_\_() (IntegerPartitions method)@\spxentry{\_\_getitem\_\_()}\spxextra{IntegerPartitions method}}

\begin{fulllineitems}
\phantomsection\label{\detokenize{index:Snob2.IntegerPartitions.__getitem__}}\pysiglinewithargsret{\sphinxbfcode{\sphinxupquote{\_\_getitem\_\_}}}{\emph{\DUrole{n}{self}\DUrole{p}{:} \DUrole{n}{{\hyperref[\detokenize{index:Snob2.IntegerPartitions}]{\sphinxcrossref{Snob2.IntegerPartitions}}}}}, \emph{\DUrole{n}{arg0}\DUrole{p}{:} \DUrole{n}{\sphinxhref{https://docs.python.org/3/library/functions.html\#int}{int}}}}{{ $\rightarrow$ {\hyperref[\detokenize{index:Snob2.IntegerPartition}]{\sphinxcrossref{Snob2.IntegerPartition}}}}}
\sphinxAtStartPar
Return the i’th integer partition.

\end{fulllineitems}

\index{\_\_init\_\_() (IntegerPartitions method)@\spxentry{\_\_init\_\_()}\spxextra{IntegerPartitions method}}

\begin{fulllineitems}
\phantomsection\label{\detokenize{index:Snob2.IntegerPartitions.__init__}}\pysiglinewithargsret{\sphinxbfcode{\sphinxupquote{\_\_init\_\_}}}{\emph{\DUrole{n}{self}\DUrole{p}{:} \DUrole{n}{{\hyperref[\detokenize{index:Snob2.IntegerPartitions}]{\sphinxcrossref{Snob2.IntegerPartitions}}}}}, \emph{\DUrole{n}{arg0}\DUrole{p}{:} \DUrole{n}{\sphinxhref{https://docs.python.org/3/library/functions.html\#int}{int}}}}{{ $\rightarrow$ \sphinxhref{https://docs.python.org/3/library/constants.html\#None}{None}}}
\sphinxAtStartPar
Create an object to represent all integer partitions of n.

\end{fulllineitems}

\index{\_\_len\_\_() (IntegerPartitions method)@\spxentry{\_\_len\_\_()}\spxextra{IntegerPartitions method}}

\begin{fulllineitems}
\phantomsection\label{\detokenize{index:Snob2.IntegerPartitions.__len__}}\pysiglinewithargsret{\sphinxbfcode{\sphinxupquote{\_\_len\_\_}}}{\emph{\DUrole{n}{self}\DUrole{p}{:} \DUrole{n}{{\hyperref[\detokenize{index:Snob2.IntegerPartitions}]{\sphinxcrossref{Snob2.IntegerPartitions}}}}}}{{ $\rightarrow$ \sphinxhref{https://docs.python.org/3/library/functions.html\#int}{int}}}
\sphinxAtStartPar
Return the number of integer partition.

\end{fulllineitems}

\index{at() (IntegerPartitions method)@\spxentry{at()}\spxextra{IntegerPartitions method}}

\begin{fulllineitems}
\phantomsection\label{\detokenize{index:Snob2.IntegerPartitions.at}}\pysiglinewithargsret{\sphinxbfcode{\sphinxupquote{at}}}{\emph{\DUrole{n}{self}\DUrole{p}{:} \DUrole{n}{{\hyperref[\detokenize{index:Snob2.IntegerPartitions}]{\sphinxcrossref{Snob2.IntegerPartitions}}}}}, \emph{\DUrole{n}{arg0}\DUrole{p}{:} \DUrole{n}{\sphinxhref{https://docs.python.org/3/library/functions.html\#int}{int}}}}{{ $\rightarrow$ {\hyperref[\detokenize{index:Snob2.IntegerPartition}]{\sphinxcrossref{Snob2.IntegerPartition}}}}}
\sphinxAtStartPar
Return the i’th integer partition.

\end{fulllineitems}


\end{fulllineitems}


\begin{DUlineblock}{0em}
\item[] 
\end{DUlineblock}
\index{YoungTableau (class in Snob2)@\spxentry{YoungTableau}\spxextra{class in Snob2}}

\begin{fulllineitems}
\phantomsection\label{\detokenize{index:Snob2.YoungTableau}}\pysigline{\sphinxbfcode{\sphinxupquote{class }}\sphinxbfcode{\sphinxupquote{YoungTableau}}}~\index{\_\_init\_\_() (YoungTableau method)@\spxentry{\_\_init\_\_()}\spxextra{YoungTableau method}}

\begin{fulllineitems}
\phantomsection\label{\detokenize{index:Snob2.YoungTableau.__init__}}\pysiglinewithargsret{\sphinxbfcode{\sphinxupquote{\_\_init\_\_}}}{\emph{\DUrole{n}{self}\DUrole{p}{:} \DUrole{n}{{\hyperref[\detokenize{index:Snob2.YoungTableau}]{\sphinxcrossref{Snob2.YoungTableau}}}}}, \emph{\DUrole{n}{arg0}\DUrole{p}{:} \DUrole{n}{{\hyperref[\detokenize{index:Snob2.IntegerPartition}]{\sphinxcrossref{Snob2.IntegerPartition}}}}}}{{ $\rightarrow$ \sphinxhref{https://docs.python.org/3/library/constants.html\#None}{None}}}
\sphinxAtStartPar
Return a tableau of the given shape filled with 1,…,n

\end{fulllineitems}

\index{\_\_str\_\_() (YoungTableau method)@\spxentry{\_\_str\_\_()}\spxextra{YoungTableau method}}

\begin{fulllineitems}
\phantomsection\label{\detokenize{index:Snob2.YoungTableau.__str__}}\pysiglinewithargsret{\sphinxbfcode{\sphinxupquote{\_\_str\_\_}}}{\emph{\DUrole{n}{self}\DUrole{p}{:} \DUrole{n}{{\hyperref[\detokenize{index:Snob2.YoungTableau}]{\sphinxcrossref{Snob2.YoungTableau}}}}}, \emph{\DUrole{n}{indent}\DUrole{p}{:} \DUrole{n}{\sphinxhref{https://docs.python.org/3/library/stdtypes.html\#str}{str}} \DUrole{o}{=} \DUrole{default_value}{\textquotesingle{}\textquotesingle{}}}}{{ $\rightarrow$ \sphinxhref{https://docs.python.org/3/library/stdtypes.html\#str}{str}}}
\sphinxAtStartPar
Print the tableau to string.

\end{fulllineitems}

\index{at() (YoungTableau method)@\spxentry{at()}\spxextra{YoungTableau method}}

\begin{fulllineitems}
\phantomsection\label{\detokenize{index:Snob2.YoungTableau.at}}\pysiglinewithargsret{\sphinxbfcode{\sphinxupquote{at}}}{\emph{\DUrole{n}{self}\DUrole{p}{:} \DUrole{n}{{\hyperref[\detokenize{index:Snob2.YoungTableau}]{\sphinxcrossref{Snob2.YoungTableau}}}}}, \emph{\DUrole{n}{arg0}\DUrole{p}{:} \DUrole{n}{\sphinxhref{https://docs.python.org/3/library/functions.html\#int}{int}}}, \emph{\DUrole{n}{arg1}\DUrole{p}{:} \DUrole{n}{\sphinxhref{https://docs.python.org/3/library/functions.html\#int}{int}}}}{{ $\rightarrow$ \sphinxhref{https://docs.python.org/3/library/functions.html\#int}{int}}}
\sphinxAtStartPar
Return the integer at position (i,j) in the tableau.

\end{fulllineitems}

\index{getk() (YoungTableau method)@\spxentry{getk()}\spxextra{YoungTableau method}}

\begin{fulllineitems}
\phantomsection\label{\detokenize{index:Snob2.YoungTableau.getk}}\pysiglinewithargsret{\sphinxbfcode{\sphinxupquote{getk}}}{\emph{\DUrole{n}{self}\DUrole{p}{:} \DUrole{n}{{\hyperref[\detokenize{index:Snob2.YoungTableau}]{\sphinxcrossref{Snob2.YoungTableau}}}}}}{{ $\rightarrow$ \sphinxhref{https://docs.python.org/3/library/functions.html\#int}{int}}}
\sphinxAtStartPar
Return the number of rows.

\end{fulllineitems}

\index{shape() (YoungTableau method)@\spxentry{shape()}\spxextra{YoungTableau method}}

\begin{fulllineitems}
\phantomsection\label{\detokenize{index:Snob2.YoungTableau.shape}}\pysiglinewithargsret{\sphinxbfcode{\sphinxupquote{shape}}}{\emph{\DUrole{n}{self}\DUrole{p}{:} \DUrole{n}{{\hyperref[\detokenize{index:Snob2.YoungTableau}]{\sphinxcrossref{Snob2.YoungTableau}}}}}}{{ $\rightarrow$ {\hyperref[\detokenize{index:Snob2.IntegerPartition}]{\sphinxcrossref{Snob2.IntegerPartition}}}}}
\sphinxAtStartPar
Return the integer partition describing the shape of this tableau.

\end{fulllineitems}


\end{fulllineitems}


\begin{DUlineblock}{0em}
\item[] 
\end{DUlineblock}
\index{Permutation (class in Snob2)@\spxentry{Permutation}\spxextra{class in Snob2}}

\begin{fulllineitems}
\phantomsection\label{\detokenize{index:Snob2.Permutation}}\pysigline{\sphinxbfcode{\sphinxupquote{class }}\sphinxbfcode{\sphinxupquote{Permutation}}}
\sphinxAtStartPar
Class to represent a permutation sigma of (1,2,…,n)
\index{\_\_eq\_\_() (Permutation method)@\spxentry{\_\_eq\_\_()}\spxextra{Permutation method}}

\begin{fulllineitems}
\phantomsection\label{\detokenize{index:Snob2.Permutation.__eq__}}\pysiglinewithargsret{\sphinxbfcode{\sphinxupquote{\_\_eq\_\_}}}{\emph{\DUrole{n}{self}\DUrole{p}{:} \DUrole{n}{{\hyperref[\detokenize{index:Snob2.Permutation}]{\sphinxcrossref{Snob2.Permutation}}}}}, \emph{\DUrole{n}{arg0}\DUrole{p}{:} \DUrole{n}{{\hyperref[\detokenize{index:Snob2.Permutation}]{\sphinxcrossref{Snob2.Permutation}}}}}}{{ $\rightarrow$ \sphinxhref{https://docs.python.org/3/library/functions.html\#bool}{bool}}}
\end{fulllineitems}

\index{\_\_getitem\_\_() (Permutation method)@\spxentry{\_\_getitem\_\_()}\spxextra{Permutation method}}

\begin{fulllineitems}
\phantomsection\label{\detokenize{index:Snob2.Permutation.__getitem__}}\pysiglinewithargsret{\sphinxbfcode{\sphinxupquote{\_\_getitem\_\_}}}{\emph{\DUrole{n}{self}\DUrole{p}{:} \DUrole{n}{{\hyperref[\detokenize{index:Snob2.Permutation}]{\sphinxcrossref{Snob2.Permutation}}}}}, \emph{\DUrole{n}{arg0}\DUrole{p}{:} \DUrole{n}{\sphinxhref{https://docs.python.org/3/library/functions.html\#int}{int}}}}{{ $\rightarrow$ \sphinxhref{https://docs.python.org/3/library/functions.html\#int}{int}}}
\sphinxAtStartPar
Return sigma(i)

\end{fulllineitems}

\index{\_\_imul\_\_() (Permutation method)@\spxentry{\_\_imul\_\_()}\spxextra{Permutation method}}

\begin{fulllineitems}
\phantomsection\label{\detokenize{index:Snob2.Permutation.__imul__}}\pysiglinewithargsret{\sphinxbfcode{\sphinxupquote{\_\_imul\_\_}}}{\emph{\DUrole{n}{self}\DUrole{p}{:} \DUrole{n}{{\hyperref[\detokenize{index:Snob2.Permutation}]{\sphinxcrossref{Snob2.Permutation}}}}}, \emph{\DUrole{n}{arg0}\DUrole{p}{:} \DUrole{n}{{\hyperref[\detokenize{index:Snob2.Permutation}]{\sphinxcrossref{Snob2.Permutation}}}}}}{{ $\rightarrow$ {\hyperref[\detokenize{index:Snob2.Permutation}]{\sphinxcrossref{Snob2.Permutation}}}}}
\end{fulllineitems}

\index{\_\_init\_\_() (Permutation method)@\spxentry{\_\_init\_\_()}\spxextra{Permutation method}}

\begin{fulllineitems}
\phantomsection\label{\detokenize{index:Snob2.Permutation.__init__}}\pysiglinewithargsret{\sphinxbfcode{\sphinxupquote{\_\_init\_\_}}}{\emph{\DUrole{o}{*}\DUrole{n}{args}}, \emph{\DUrole{o}{**}\DUrole{n}{kwargs}}}{}
\sphinxAtStartPar
Overloaded function.
\begin{enumerate}
\sphinxsetlistlabels{\arabic}{enumi}{enumii}{}{.}%
\item {} 
\sphinxAtStartPar
\_\_init\_\_(self: Snob2.Permutation, arg0: int) \sphinxhyphen{}\textgreater{} None

\item {} 
\sphinxAtStartPar
\_\_init\_\_(self: Snob2.Permutation, arg0: int, arg1: cnine::fill\_raw) \sphinxhyphen{}\textgreater{} None

\item {} 
\sphinxAtStartPar
\_\_init\_\_(self: Snob2.Permutation, arg0: int, arg1: cnine::fill\_identity) \sphinxhyphen{}\textgreater{} None

\end{enumerate}

\sphinxAtStartPar
Return the identity permutation on n items
\begin{enumerate}
\sphinxsetlistlabels{\arabic}{enumi}{enumii}{}{.}%
\setcounter{enumi}{3}
\item {} 
\sphinxAtStartPar
\_\_init\_\_(self: Snob2.Permutation, arg0: List{[}int{]}) \sphinxhyphen{}\textgreater{} None

\end{enumerate}

\sphinxAtStartPar
Initialize the permutation from a list.

\end{fulllineitems}

\index{\_\_inv\_\_() (Permutation method)@\spxentry{\_\_inv\_\_()}\spxextra{Permutation method}}

\begin{fulllineitems}
\phantomsection\label{\detokenize{index:Snob2.Permutation.__inv__}}\pysiglinewithargsret{\sphinxbfcode{\sphinxupquote{\_\_inv\_\_}}}{\emph{\DUrole{n}{self}\DUrole{p}{:} \DUrole{n}{{\hyperref[\detokenize{index:Snob2.Permutation}]{\sphinxcrossref{Snob2.Permutation}}}}}}{{ $\rightarrow$ {\hyperref[\detokenize{index:Snob2.Permutation}]{\sphinxcrossref{Snob2.Permutation}}}}}
\end{fulllineitems}

\index{\_\_mul\_\_() (Permutation method)@\spxentry{\_\_mul\_\_()}\spxextra{Permutation method}}

\begin{fulllineitems}
\phantomsection\label{\detokenize{index:Snob2.Permutation.__mul__}}\pysiglinewithargsret{\sphinxbfcode{\sphinxupquote{\_\_mul\_\_}}}{\emph{\DUrole{n}{self}\DUrole{p}{:} \DUrole{n}{{\hyperref[\detokenize{index:Snob2.Permutation}]{\sphinxcrossref{Snob2.Permutation}}}}}, \emph{\DUrole{n}{arg0}\DUrole{p}{:} \DUrole{n}{{\hyperref[\detokenize{index:Snob2.Permutation}]{\sphinxcrossref{Snob2.Permutation}}}}}}{{ $\rightarrow$ {\hyperref[\detokenize{index:Snob2.Permutation}]{\sphinxcrossref{Snob2.Permutation}}}}}
\end{fulllineitems}

\index{\_\_str\_\_() (Permutation method)@\spxentry{\_\_str\_\_()}\spxextra{Permutation method}}

\begin{fulllineitems}
\phantomsection\label{\detokenize{index:Snob2.Permutation.__str__}}\pysiglinewithargsret{\sphinxbfcode{\sphinxupquote{\_\_str\_\_}}}{\emph{\DUrole{n}{self}\DUrole{p}{:} \DUrole{n}{{\hyperref[\detokenize{index:Snob2.Permutation}]{\sphinxcrossref{Snob2.Permutation}}}}}, \emph{\DUrole{n}{indent}\DUrole{p}{:} \DUrole{n}{{\hyperref[\detokenize{index:Snob2.Permutation.str}]{\sphinxcrossref{str}}}} \DUrole{o}{=} \DUrole{default_value}{\textquotesingle{}\textquotesingle{}}}}{{ $\rightarrow$ {\hyperref[\detokenize{index:Snob2.Permutation.str}]{\sphinxcrossref{str}}}}}
\end{fulllineitems}

\index{getn() (Permutation method)@\spxentry{getn()}\spxextra{Permutation method}}

\begin{fulllineitems}
\phantomsection\label{\detokenize{index:Snob2.Permutation.getn}}\pysiglinewithargsret{\sphinxbfcode{\sphinxupquote{getn}}}{\emph{\DUrole{n}{self}\DUrole{p}{:} \DUrole{n}{{\hyperref[\detokenize{index:Snob2.Permutation}]{\sphinxcrossref{Snob2.Permutation}}}}}}{{ $\rightarrow$ \sphinxhref{https://docs.python.org/3/library/functions.html\#int}{int}}}
\sphinxAtStartPar
Return n.

\end{fulllineitems}

\index{identity() (Permutation static method)@\spxentry{identity()}\spxextra{Permutation static method}}

\begin{fulllineitems}
\phantomsection\label{\detokenize{index:Snob2.Permutation.identity}}\pysiglinewithargsret{\sphinxbfcode{\sphinxupquote{static }}\sphinxbfcode{\sphinxupquote{identity}}}{\emph{\DUrole{n}{arg0}\DUrole{p}{:} \DUrole{n}{\sphinxhref{https://docs.python.org/3/library/functions.html\#int}{int}}}}{{ $\rightarrow$ {\hyperref[\detokenize{index:Snob2.Permutation}]{\sphinxcrossref{Snob2.Permutation}}}}}
\sphinxAtStartPar
Return the identity permutation of n.

\end{fulllineitems}

\index{inv() (Permutation method)@\spxentry{inv()}\spxextra{Permutation method}}

\begin{fulllineitems}
\phantomsection\label{\detokenize{index:Snob2.Permutation.inv}}\pysiglinewithargsret{\sphinxbfcode{\sphinxupquote{inv}}}{\emph{\DUrole{n}{self}\DUrole{p}{:} \DUrole{n}{{\hyperref[\detokenize{index:Snob2.Permutation}]{\sphinxcrossref{Snob2.Permutation}}}}}}{{ $\rightarrow$ {\hyperref[\detokenize{index:Snob2.Permutation}]{\sphinxcrossref{Snob2.Permutation}}}}}
\sphinxAtStartPar
Return the inverse permutation.

\end{fulllineitems}

\index{str() (Permutation method)@\spxentry{str()}\spxextra{Permutation method}}

\begin{fulllineitems}
\phantomsection\label{\detokenize{index:Snob2.Permutation.str}}\pysiglinewithargsret{\sphinxbfcode{\sphinxupquote{str}}}{\emph{\DUrole{n}{self}\DUrole{p}{:} \DUrole{n}{{\hyperref[\detokenize{index:Snob2.Permutation}]{\sphinxcrossref{Snob2.Permutation}}}}}, \emph{\DUrole{n}{indent}\DUrole{p}{:} \DUrole{n}{{\hyperref[\detokenize{index:Snob2.Permutation.str}]{\sphinxcrossref{str}}}} \DUrole{o}{=} \DUrole{default_value}{\textquotesingle{}\textquotesingle{}}}}{{ $\rightarrow$ {\hyperref[\detokenize{index:Snob2.Permutation.str}]{\sphinxcrossref{str}}}}}
\end{fulllineitems}


\end{fulllineitems}


\begin{DUlineblock}{0em}
\item[] 
\end{DUlineblock}


\section{Symmetric group classes}
\label{\detokenize{index:id4}}
\begin{DUlineblock}{0em}
\item[] 
\end{DUlineblock}
\index{Sn (class in Snob2)@\spxentry{Sn}\spxextra{class in Snob2}}

\begin{fulllineitems}
\phantomsection\label{\detokenize{index:Snob2.Sn}}\pysiglinewithargsret{\sphinxbfcode{\sphinxupquote{class }}\sphinxbfcode{\sphinxupquote{Sn}}}{\emph{\DUrole{n}{n}}}{}~\index{\_\_getitem\_\_() (Sn method)@\spxentry{\_\_getitem\_\_()}\spxextra{Sn method}}

\begin{fulllineitems}
\phantomsection\label{\detokenize{index:Snob2.Sn.__getitem__}}\pysiglinewithargsret{\sphinxbfcode{\sphinxupquote{\_\_getitem\_\_}}}{\emph{\DUrole{n}{self}\DUrole{p}{:} \DUrole{n}{{\hyperref[\detokenize{index:Snob2.Sn}]{\sphinxcrossref{Snob2.Sn}}}}}, \emph{\DUrole{n}{arg0}\DUrole{p}{:} \DUrole{n}{\sphinxhref{https://docs.python.org/3/library/functions.html\#int}{int}}}}{{ $\rightarrow$ {\hyperref[\detokenize{index:Snob2.SnElement}]{\sphinxcrossref{Snob2.SnElement}}}}}
\end{fulllineitems}

\index{\_\_init\_\_() (Sn method)@\spxentry{\_\_init\_\_()}\spxextra{Sn method}}

\begin{fulllineitems}
\phantomsection\label{\detokenize{index:Snob2.Sn.__init__}}\pysiglinewithargsret{\sphinxbfcode{\sphinxupquote{\_\_init\_\_}}}{\emph{\DUrole{n}{self}\DUrole{p}{:} \DUrole{n}{{\hyperref[\detokenize{index:Snob2.Sn}]{\sphinxcrossref{Snob2.Sn}}}}}, \emph{\DUrole{n}{arg0}\DUrole{p}{:} \DUrole{n}{\sphinxhref{https://docs.python.org/3/library/functions.html\#int}{int}}}}{{ $\rightarrow$ \sphinxhref{https://docs.python.org/3/library/constants.html\#None}{None}}}
\end{fulllineitems}

\index{\_\_len\_\_() (Sn method)@\spxentry{\_\_len\_\_()}\spxextra{Sn method}}

\begin{fulllineitems}
\phantomsection\label{\detokenize{index:Snob2.Sn.__len__}}\pysiglinewithargsret{\sphinxbfcode{\sphinxupquote{\_\_len\_\_}}}{\emph{\DUrole{n}{self}\DUrole{p}{:} \DUrole{n}{{\hyperref[\detokenize{index:Snob2.Sn}]{\sphinxcrossref{Snob2.Sn}}}}}}{{ $\rightarrow$ \sphinxhref{https://docs.python.org/3/library/functions.html\#int}{int}}}
\end{fulllineitems}

\index{\_\_str\_\_() (Sn method)@\spxentry{\_\_str\_\_()}\spxextra{Sn method}}

\begin{fulllineitems}
\phantomsection\label{\detokenize{index:Snob2.Sn.__str__}}\pysiglinewithargsret{\sphinxbfcode{\sphinxupquote{\_\_str\_\_}}}{\emph{\DUrole{n}{self}\DUrole{p}{:} \DUrole{n}{{\hyperref[\detokenize{index:Snob2.Sn}]{\sphinxcrossref{Snob2.Sn}}}}}, \emph{\DUrole{n}{indent}\DUrole{p}{:} \DUrole{n}{{\hyperref[\detokenize{index:Snob2.Sn.str}]{\sphinxcrossref{str}}}} \DUrole{o}{=} \DUrole{default_value}{\textquotesingle{}\textquotesingle{}}}}{{ $\rightarrow$ {\hyperref[\detokenize{index:Snob2.Sn.str}]{\sphinxcrossref{str}}}}}
\end{fulllineitems}

\index{cclass() (Sn method)@\spxentry{cclass()}\spxextra{Sn method}}

\begin{fulllineitems}
\phantomsection\label{\detokenize{index:Snob2.Sn.cclass}}\pysiglinewithargsret{\sphinxbfcode{\sphinxupquote{cclass}}}{\emph{\DUrole{o}{*}\DUrole{n}{args}}, \emph{\DUrole{o}{**}\DUrole{n}{kwargs}}}{}
\sphinxAtStartPar
Overloaded function.
\begin{enumerate}
\sphinxsetlistlabels{\arabic}{enumi}{enumii}{}{.}%
\item {} 
\sphinxAtStartPar
cclass(self: Snob2.Sn, arg0: int) \sphinxhyphen{}\textgreater{} Snob2.SnCClass

\item {} 
\sphinxAtStartPar
cclass(self: Snob2.Sn, arg0: Snob2.IntegerPartition) \sphinxhyphen{}\textgreater{} Snob2.SnCClass

\end{enumerate}

\end{fulllineitems}

\index{cclass\_size() (Sn method)@\spxentry{cclass\_size()}\spxextra{Sn method}}

\begin{fulllineitems}
\phantomsection\label{\detokenize{index:Snob2.Sn.cclass_size}}\pysiglinewithargsret{\sphinxbfcode{\sphinxupquote{cclass\_size}}}{\emph{\DUrole{o}{*}\DUrole{n}{args}}, \emph{\DUrole{o}{**}\DUrole{n}{kwargs}}}{}
\sphinxAtStartPar
Overloaded function.
\begin{enumerate}
\sphinxsetlistlabels{\arabic}{enumi}{enumii}{}{.}%
\item {} 
\sphinxAtStartPar
cclass\_size(self: Snob2.Sn, arg0: Snob2.IntegerPartition) \sphinxhyphen{}\textgreater{} int

\item {} 
\sphinxAtStartPar
cclass\_size(self: Snob2.Sn, arg0: Snob2.SnCClass) \sphinxhyphen{}\textgreater{} int

\end{enumerate}

\end{fulllineitems}

\index{character() (Sn method)@\spxentry{character()}\spxextra{Sn method}}

\begin{fulllineitems}
\phantomsection\label{\detokenize{index:Snob2.Sn.character}}\pysiglinewithargsret{\sphinxbfcode{\sphinxupquote{character}}}{\emph{\DUrole{n}{self}\DUrole{p}{:} \DUrole{n}{{\hyperref[\detokenize{index:Snob2.Sn}]{\sphinxcrossref{Snob2.Sn}}}}}, \emph{\DUrole{n}{arg0}\DUrole{p}{:} \DUrole{n}{{\hyperref[\detokenize{index:Snob2.IntegerPartition}]{\sphinxcrossref{Snob2.IntegerPartition}}}}}}{{ $\rightarrow$ Snob2.SnClassFunction}}
\end{fulllineitems}

\index{element() (Sn method)@\spxentry{element()}\spxextra{Sn method}}

\begin{fulllineitems}
\phantomsection\label{\detokenize{index:Snob2.Sn.element}}\pysiglinewithargsret{\sphinxbfcode{\sphinxupquote{element}}}{\emph{\DUrole{n}{self}\DUrole{p}{:} \DUrole{n}{{\hyperref[\detokenize{index:Snob2.Sn}]{\sphinxcrossref{Snob2.Sn}}}}}, \emph{\DUrole{n}{arg0}\DUrole{p}{:} \DUrole{n}{\sphinxhref{https://docs.python.org/3/library/functions.html\#int}{int}}}}{{ $\rightarrow$ {\hyperref[\detokenize{index:Snob2.SnElement}]{\sphinxcrossref{Snob2.SnElement}}}}}
\end{fulllineitems}

\index{getn() (Sn method)@\spxentry{getn()}\spxextra{Sn method}}

\begin{fulllineitems}
\phantomsection\label{\detokenize{index:Snob2.Sn.getn}}\pysiglinewithargsret{\sphinxbfcode{\sphinxupquote{getn}}}{\emph{\DUrole{n}{self}\DUrole{p}{:} \DUrole{n}{{\hyperref[\detokenize{index:Snob2.Sn}]{\sphinxcrossref{Snob2.Sn}}}}}}{{ $\rightarrow$ \sphinxhref{https://docs.python.org/3/library/functions.html\#int}{int}}}
\end{fulllineitems}

\index{identity() (Sn method)@\spxentry{identity()}\spxextra{Sn method}}

\begin{fulllineitems}
\phantomsection\label{\detokenize{index:Snob2.Sn.identity}}\pysiglinewithargsret{\sphinxbfcode{\sphinxupquote{identity}}}{\emph{\DUrole{n}{self}\DUrole{p}{:} \DUrole{n}{{\hyperref[\detokenize{index:Snob2.Sn}]{\sphinxcrossref{Snob2.Sn}}}}}}{{ $\rightarrow$ {\hyperref[\detokenize{index:Snob2.SnElement}]{\sphinxcrossref{Snob2.SnElement}}}}}
\end{fulllineitems}

\index{index() (Sn method)@\spxentry{index()}\spxextra{Sn method}}

\begin{fulllineitems}
\phantomsection\label{\detokenize{index:Snob2.Sn.index}}\pysiglinewithargsret{\sphinxbfcode{\sphinxupquote{index}}}{\emph{\DUrole{o}{*}\DUrole{n}{args}}, \emph{\DUrole{o}{**}\DUrole{n}{kwargs}}}{}
\sphinxAtStartPar
Overloaded function.
\begin{enumerate}
\sphinxsetlistlabels{\arabic}{enumi}{enumii}{}{.}%
\item {} 
\sphinxAtStartPar
index(self: Snob2.Sn, arg0: Snob2.SnElement) \sphinxhyphen{}\textgreater{} int

\item {} 
\sphinxAtStartPar
index(self: Snob2.Sn, arg0: Snob2.IntegerPartition) \sphinxhyphen{}\textgreater{} int

\item {} 
\sphinxAtStartPar
index(self: Snob2.Sn, arg0: Snob2.SnCClass) \sphinxhyphen{}\textgreater{} int

\end{enumerate}

\end{fulllineitems}

\index{irrep() (Sn method)@\spxentry{irrep()}\spxextra{Sn method}}

\begin{fulllineitems}
\phantomsection\label{\detokenize{index:Snob2.Sn.irrep}}\pysiglinewithargsret{\sphinxbfcode{\sphinxupquote{irrep}}}{\emph{\DUrole{n}{self}\DUrole{p}{:} \DUrole{n}{{\hyperref[\detokenize{index:Snob2.Sn}]{\sphinxcrossref{Snob2.Sn}}}}}, \emph{\DUrole{n}{arg0}\DUrole{p}{:} \DUrole{n}{{\hyperref[\detokenize{index:Snob2.IntegerPartition}]{\sphinxcrossref{Snob2.IntegerPartition}}}}}}{{ $\rightarrow$ {\hyperref[\detokenize{index:Snob2.SnIrrep}]{\sphinxcrossref{Snob2.SnIrrep}}}}}
\end{fulllineitems}

\index{ncclasses() (Sn method)@\spxentry{ncclasses()}\spxextra{Sn method}}

\begin{fulllineitems}
\phantomsection\label{\detokenize{index:Snob2.Sn.ncclasses}}\pysiglinewithargsret{\sphinxbfcode{\sphinxupquote{ncclasses}}}{\emph{\DUrole{n}{self}\DUrole{p}{:} \DUrole{n}{{\hyperref[\detokenize{index:Snob2.Sn}]{\sphinxcrossref{Snob2.Sn}}}}}}{{ $\rightarrow$ \sphinxhref{https://docs.python.org/3/library/functions.html\#int}{int}}}
\end{fulllineitems}

\index{nchars() (Sn method)@\spxentry{nchars()}\spxextra{Sn method}}

\begin{fulllineitems}
\phantomsection\label{\detokenize{index:Snob2.Sn.nchars}}\pysiglinewithargsret{\sphinxbfcode{\sphinxupquote{nchars}}}{\emph{\DUrole{n}{self}\DUrole{p}{:} \DUrole{n}{{\hyperref[\detokenize{index:Snob2.Sn}]{\sphinxcrossref{Snob2.Sn}}}}}}{{ $\rightarrow$ \sphinxhref{https://docs.python.org/3/library/functions.html\#int}{int}}}
\end{fulllineitems}

\index{order() (Sn method)@\spxentry{order()}\spxextra{Sn method}}

\begin{fulllineitems}
\phantomsection\label{\detokenize{index:Snob2.Sn.order}}\pysiglinewithargsret{\sphinxbfcode{\sphinxupquote{order}}}{\emph{\DUrole{n}{self}\DUrole{p}{:} \DUrole{n}{{\hyperref[\detokenize{index:Snob2.Sn}]{\sphinxcrossref{Snob2.Sn}}}}}}{{ $\rightarrow$ \sphinxhref{https://docs.python.org/3/library/functions.html\#int}{int}}}
\end{fulllineitems}

\index{random() (Sn method)@\spxentry{random()}\spxextra{Sn method}}

\begin{fulllineitems}
\phantomsection\label{\detokenize{index:Snob2.Sn.random}}\pysiglinewithargsret{\sphinxbfcode{\sphinxupquote{random}}}{\emph{\DUrole{n}{self}\DUrole{p}{:} \DUrole{n}{{\hyperref[\detokenize{index:Snob2.Sn}]{\sphinxcrossref{Snob2.Sn}}}}}}{{ $\rightarrow$ {\hyperref[\detokenize{index:Snob2.SnElement}]{\sphinxcrossref{Snob2.SnElement}}}}}
\end{fulllineitems}

\index{str() (Sn method)@\spxentry{str()}\spxextra{Sn method}}

\begin{fulllineitems}
\phantomsection\label{\detokenize{index:Snob2.Sn.str}}\pysiglinewithargsret{\sphinxbfcode{\sphinxupquote{str}}}{\emph{\DUrole{n}{self}\DUrole{p}{:} \DUrole{n}{{\hyperref[\detokenize{index:Snob2.Sn}]{\sphinxcrossref{Snob2.Sn}}}}}, \emph{\DUrole{n}{indent}\DUrole{p}{:} \DUrole{n}{{\hyperref[\detokenize{index:Snob2.Sn.str}]{\sphinxcrossref{str}}}} \DUrole{o}{=} \DUrole{default_value}{\textquotesingle{}\textquotesingle{}}}}{{ $\rightarrow$ {\hyperref[\detokenize{index:Snob2.Sn.str}]{\sphinxcrossref{str}}}}}
\end{fulllineitems}


\end{fulllineitems}


\begin{DUlineblock}{0em}
\item[] 
\item[] 
\end{DUlineblock}
\index{SnElement (class in Snob2)@\spxentry{SnElement}\spxextra{class in Snob2}}

\begin{fulllineitems}
\phantomsection\label{\detokenize{index:Snob2.SnElement}}\pysigline{\sphinxbfcode{\sphinxupquote{class }}\sphinxbfcode{\sphinxupquote{SnElement}}}
\sphinxAtStartPar
Class to represent symmetric group elements.
\index{\_\_eq\_\_() (SnElement method)@\spxentry{\_\_eq\_\_()}\spxextra{SnElement method}}

\begin{fulllineitems}
\phantomsection\label{\detokenize{index:Snob2.SnElement.__eq__}}\pysiglinewithargsret{\sphinxbfcode{\sphinxupquote{\_\_eq\_\_}}}{\emph{\DUrole{n}{self}\DUrole{p}{:} \DUrole{n}{{\hyperref[\detokenize{index:Snob2.SnElement}]{\sphinxcrossref{Snob2.SnElement}}}}}, \emph{\DUrole{n}{arg0}\DUrole{p}{:} \DUrole{n}{{\hyperref[\detokenize{index:Snob2.Permutation}]{\sphinxcrossref{Snob2.Permutation}}}}}}{{ $\rightarrow$ \sphinxhref{https://docs.python.org/3/library/functions.html\#bool}{bool}}}
\end{fulllineitems}

\index{\_\_getitem\_\_() (SnElement method)@\spxentry{\_\_getitem\_\_()}\spxextra{SnElement method}}

\begin{fulllineitems}
\phantomsection\label{\detokenize{index:Snob2.SnElement.__getitem__}}\pysiglinewithargsret{\sphinxbfcode{\sphinxupquote{\_\_getitem\_\_}}}{\emph{\DUrole{n}{self}\DUrole{p}{:} \DUrole{n}{{\hyperref[\detokenize{index:Snob2.SnElement}]{\sphinxcrossref{Snob2.SnElement}}}}}, \emph{\DUrole{n}{arg0}\DUrole{p}{:} \DUrole{n}{\sphinxhref{https://docs.python.org/3/library/functions.html\#int}{int}}}}{{ $\rightarrow$ \sphinxhref{https://docs.python.org/3/library/functions.html\#int}{int}}}
\end{fulllineitems}

\index{\_\_imul\_\_() (SnElement method)@\spxentry{\_\_imul\_\_()}\spxextra{SnElement method}}

\begin{fulllineitems}
\phantomsection\label{\detokenize{index:Snob2.SnElement.__imul__}}\pysiglinewithargsret{\sphinxbfcode{\sphinxupquote{\_\_imul\_\_}}}{\emph{\DUrole{n}{self}\DUrole{p}{:} \DUrole{n}{{\hyperref[\detokenize{index:Snob2.SnElement}]{\sphinxcrossref{Snob2.SnElement}}}}}, \emph{\DUrole{n}{arg0}\DUrole{p}{:} \DUrole{n}{{\hyperref[\detokenize{index:Snob2.Permutation}]{\sphinxcrossref{Snob2.Permutation}}}}}}{{ $\rightarrow$ {\hyperref[\detokenize{index:Snob2.Permutation}]{\sphinxcrossref{Snob2.Permutation}}}}}
\end{fulllineitems}

\index{\_\_init\_\_() (SnElement method)@\spxentry{\_\_init\_\_()}\spxextra{SnElement method}}

\begin{fulllineitems}
\phantomsection\label{\detokenize{index:Snob2.SnElement.__init__}}\pysiglinewithargsret{\sphinxbfcode{\sphinxupquote{\_\_init\_\_}}}{\emph{\DUrole{o}{*}\DUrole{n}{args}}, \emph{\DUrole{o}{**}\DUrole{n}{kwargs}}}{}
\sphinxAtStartPar
Overloaded function.
\begin{enumerate}
\sphinxsetlistlabels{\arabic}{enumi}{enumii}{}{.}%
\item {} 
\sphinxAtStartPar
\_\_init\_\_(self: Snob2.SnElement, arg0: int) \sphinxhyphen{}\textgreater{} None

\item {} 
\sphinxAtStartPar
\_\_init\_\_(self: Snob2.SnElement, arg0: int, arg1: cnine::fill\_raw) \sphinxhyphen{}\textgreater{} None

\item {} 
\sphinxAtStartPar
\_\_init\_\_(self: Snob2.SnElement, arg0: int, arg1: cnine::fill\_identity) \sphinxhyphen{}\textgreater{} None

\item {} 
\sphinxAtStartPar
\_\_init\_\_(self: Snob2.SnElement, arg0: List{[}int{]}) \sphinxhyphen{}\textgreater{} None

\item {} 
\sphinxAtStartPar
\_\_init\_\_(self: Snob2.SnElement, arg0: Snob2.Permutation) \sphinxhyphen{}\textgreater{} None

\end{enumerate}

\end{fulllineitems}

\index{\_\_inv\_\_() (SnElement method)@\spxentry{\_\_inv\_\_()}\spxextra{SnElement method}}

\begin{fulllineitems}
\phantomsection\label{\detokenize{index:Snob2.SnElement.__inv__}}\pysiglinewithargsret{\sphinxbfcode{\sphinxupquote{\_\_inv\_\_}}}{\emph{\DUrole{n}{self}\DUrole{p}{:} \DUrole{n}{{\hyperref[\detokenize{index:Snob2.SnElement}]{\sphinxcrossref{Snob2.SnElement}}}}}}{{ $\rightarrow$ {\hyperref[\detokenize{index:Snob2.SnElement}]{\sphinxcrossref{Snob2.SnElement}}}}}
\end{fulllineitems}

\index{\_\_mul\_\_() (SnElement method)@\spxentry{\_\_mul\_\_()}\spxextra{SnElement method}}

\begin{fulllineitems}
\phantomsection\label{\detokenize{index:Snob2.SnElement.__mul__}}\pysiglinewithargsret{\sphinxbfcode{\sphinxupquote{\_\_mul\_\_}}}{\emph{\DUrole{n}{self}\DUrole{p}{:} \DUrole{n}{{\hyperref[\detokenize{index:Snob2.SnElement}]{\sphinxcrossref{Snob2.SnElement}}}}}, \emph{\DUrole{n}{arg0}\DUrole{p}{:} \DUrole{n}{{\hyperref[\detokenize{index:Snob2.SnElement}]{\sphinxcrossref{Snob2.SnElement}}}}}}{{ $\rightarrow$ {\hyperref[\detokenize{index:Snob2.SnElement}]{\sphinxcrossref{Snob2.SnElement}}}}}
\end{fulllineitems}

\index{\_\_str\_\_() (SnElement method)@\spxentry{\_\_str\_\_()}\spxextra{SnElement method}}

\begin{fulllineitems}
\phantomsection\label{\detokenize{index:Snob2.SnElement.__str__}}\pysiglinewithargsret{\sphinxbfcode{\sphinxupquote{\_\_str\_\_}}}{\emph{\DUrole{n}{self}\DUrole{p}{:} \DUrole{n}{{\hyperref[\detokenize{index:Snob2.SnElement}]{\sphinxcrossref{Snob2.SnElement}}}}}, \emph{\DUrole{n}{indent}\DUrole{p}{:} \DUrole{n}{\sphinxhref{https://docs.python.org/3/library/stdtypes.html\#str}{str}} \DUrole{o}{=} \DUrole{default_value}{\textquotesingle{}\textquotesingle{}}}}{{ $\rightarrow$ \sphinxhref{https://docs.python.org/3/library/stdtypes.html\#str}{str}}}
\end{fulllineitems}

\index{getn() (SnElement method)@\spxentry{getn()}\spxextra{SnElement method}}

\begin{fulllineitems}
\phantomsection\label{\detokenize{index:Snob2.SnElement.getn}}\pysiglinewithargsret{\sphinxbfcode{\sphinxupquote{getn}}}{\emph{\DUrole{n}{self}\DUrole{p}{:} \DUrole{n}{{\hyperref[\detokenize{index:Snob2.SnElement}]{\sphinxcrossref{Snob2.SnElement}}}}}}{{ $\rightarrow$ \sphinxhref{https://docs.python.org/3/library/functions.html\#int}{int}}}
\sphinxAtStartPar
Return n.

\end{fulllineitems}

\index{identity() (SnElement static method)@\spxentry{identity()}\spxextra{SnElement static method}}

\begin{fulllineitems}
\phantomsection\label{\detokenize{index:Snob2.SnElement.identity}}\pysiglinewithargsret{\sphinxbfcode{\sphinxupquote{static }}\sphinxbfcode{\sphinxupquote{identity}}}{\emph{\DUrole{n}{arg0}\DUrole{p}{:} \DUrole{n}{\sphinxhref{https://docs.python.org/3/library/functions.html\#int}{int}}}}{{ $\rightarrow$ {\hyperref[\detokenize{index:Snob2.SnElement}]{\sphinxcrossref{Snob2.SnElement}}}}}
\sphinxAtStartPar
Return the identity element of Sn.

\end{fulllineitems}

\index{inv() (SnElement method)@\spxentry{inv()}\spxextra{SnElement method}}

\begin{fulllineitems}
\phantomsection\label{\detokenize{index:Snob2.SnElement.inv}}\pysiglinewithargsret{\sphinxbfcode{\sphinxupquote{inv}}}{\emph{\DUrole{n}{self}\DUrole{p}{:} \DUrole{n}{{\hyperref[\detokenize{index:Snob2.SnElement}]{\sphinxcrossref{Snob2.SnElement}}}}}}{{ $\rightarrow$ {\hyperref[\detokenize{index:Snob2.SnElement}]{\sphinxcrossref{Snob2.SnElement}}}}}
\end{fulllineitems}


\end{fulllineitems}


\begin{DUlineblock}{0em}
\item[] 
\item[] 
\end{DUlineblock}
\index{SnCharacter (class in Snob2)@\spxentry{SnCharacter}\spxextra{class in Snob2}}

\begin{fulllineitems}
\phantomsection\label{\detokenize{index:Snob2.SnCharacter}}\pysigline{\sphinxbfcode{\sphinxupquote{class }}\sphinxbfcode{\sphinxupquote{SnCharacter}}}
\sphinxAtStartPar
Class to represent characters of Sn.
\index{\_\_init\_\_() (SnCharacter method)@\spxentry{\_\_init\_\_()}\spxextra{SnCharacter method}}

\begin{fulllineitems}
\phantomsection\label{\detokenize{index:Snob2.SnCharacter.__init__}}\pysiglinewithargsret{\sphinxbfcode{\sphinxupquote{\_\_init\_\_}}}{\emph{\DUrole{n}{self}\DUrole{p}{:} \DUrole{n}{{\hyperref[\detokenize{index:Snob2.SnCharacter}]{\sphinxcrossref{Snob2.SnCharacter}}}}}, \emph{\DUrole{n}{arg0}\DUrole{p}{:} \DUrole{n}{{\hyperref[\detokenize{index:Snob2.IntegerPartition}]{\sphinxcrossref{Snob2.IntegerPartition}}}}}}{{ $\rightarrow$ \sphinxhref{https://docs.python.org/3/library/constants.html\#None}{None}}}
\sphinxAtStartPar
The character corresponding to the integer partition lambda.

\end{fulllineitems}

\index{\_\_str\_\_() (SnCharacter method)@\spxentry{\_\_str\_\_()}\spxextra{SnCharacter method}}

\begin{fulllineitems}
\phantomsection\label{\detokenize{index:Snob2.SnCharacter.__str__}}\pysiglinewithargsret{\sphinxbfcode{\sphinxupquote{\_\_str\_\_}}}{\emph{\DUrole{n}{self}\DUrole{p}{:} \DUrole{n}{{\hyperref[\detokenize{index:Snob2.SnCharacter}]{\sphinxcrossref{Snob2.SnCharacter}}}}}, \emph{\DUrole{n}{indent}\DUrole{p}{:} \DUrole{n}{\sphinxhref{https://docs.python.org/3/library/stdtypes.html\#str}{str}} \DUrole{o}{=} \DUrole{default_value}{\textquotesingle{}\textquotesingle{}}}}{{ $\rightarrow$ \sphinxhref{https://docs.python.org/3/library/stdtypes.html\#str}{str}}}
\end{fulllineitems}


\end{fulllineitems}


\begin{DUlineblock}{0em}
\item[] 
\item[] 
\end{DUlineblock}
\index{SnIrrep (class in Snob2)@\spxentry{SnIrrep}\spxextra{class in Snob2}}

\begin{fulllineitems}
\phantomsection\label{\detokenize{index:Snob2.SnIrrep}}\pysigline{\sphinxbfcode{\sphinxupquote{class }}\sphinxbfcode{\sphinxupquote{SnIrrep}}}
\sphinxAtStartPar
Class to represent the irreps of Sn.
\index{\_\_getitem\_\_() (SnIrrep method)@\spxentry{\_\_getitem\_\_()}\spxextra{SnIrrep method}}

\begin{fulllineitems}
\phantomsection\label{\detokenize{index:Snob2.SnIrrep.__getitem__}}\pysiglinewithargsret{\sphinxbfcode{\sphinxupquote{\_\_getitem\_\_}}}{\emph{\DUrole{n}{self}\DUrole{p}{:} \DUrole{n}{{\hyperref[\detokenize{index:Snob2.SnIrrep}]{\sphinxcrossref{Snob2.SnIrrep}}}}}, \emph{\DUrole{n}{arg0}\DUrole{p}{:} \DUrole{n}{{\hyperref[\detokenize{index:Snob2.SnElement}]{\sphinxcrossref{Snob2.SnElement}}}}}}{{ $\rightarrow$ cnine::RtensorObj}}
\end{fulllineitems}

\index{\_\_init\_\_() (SnIrrep method)@\spxentry{\_\_init\_\_()}\spxextra{SnIrrep method}}

\begin{fulllineitems}
\phantomsection\label{\detokenize{index:Snob2.SnIrrep.__init__}}\pysiglinewithargsret{\sphinxbfcode{\sphinxupquote{\_\_init\_\_}}}{\emph{\DUrole{o}{*}\DUrole{n}{args}}, \emph{\DUrole{o}{**}\DUrole{n}{kwargs}}}{}
\sphinxAtStartPar
Overloaded function.
\begin{enumerate}
\sphinxsetlistlabels{\arabic}{enumi}{enumii}{}{.}%
\item {} 
\sphinxAtStartPar
\_\_init\_\_(self: Snob2.SnIrrep, arg0: int) \sphinxhyphen{}\textgreater{} None

\item {} 
\sphinxAtStartPar
\_\_init\_\_(self: Snob2.SnIrrep, arg0: Snob2.IntegerPartition) \sphinxhyphen{}\textgreater{} None

\end{enumerate}

\end{fulllineitems}

\index{\_\_lt\_\_() (SnIrrep method)@\spxentry{\_\_lt\_\_()}\spxextra{SnIrrep method}}

\begin{fulllineitems}
\phantomsection\label{\detokenize{index:Snob2.SnIrrep.__lt__}}\pysiglinewithargsret{\sphinxbfcode{\sphinxupquote{\_\_lt\_\_}}}{\emph{\DUrole{n}{self}\DUrole{p}{:} \DUrole{n}{{\hyperref[\detokenize{index:Snob2.SnIrrep}]{\sphinxcrossref{Snob2.SnIrrep}}}}}, \emph{\DUrole{n}{arg0}\DUrole{p}{:} \DUrole{n}{{\hyperref[\detokenize{index:Snob2.SnIrrep}]{\sphinxcrossref{Snob2.SnIrrep}}}}}}{{ $\rightarrow$ \sphinxhref{https://docs.python.org/3/library/functions.html\#bool}{bool}}}
\end{fulllineitems}

\index{\_\_str\_\_() (SnIrrep method)@\spxentry{\_\_str\_\_()}\spxextra{SnIrrep method}}

\begin{fulllineitems}
\phantomsection\label{\detokenize{index:Snob2.SnIrrep.__str__}}\pysiglinewithargsret{\sphinxbfcode{\sphinxupquote{\_\_str\_\_}}}{\emph{\DUrole{n}{self}\DUrole{p}{:} \DUrole{n}{{\hyperref[\detokenize{index:Snob2.SnIrrep}]{\sphinxcrossref{Snob2.SnIrrep}}}}}, \emph{\DUrole{n}{indent}\DUrole{p}{:} \DUrole{n}{{\hyperref[\detokenize{index:Snob2.SnIrrep.str}]{\sphinxcrossref{str}}}} \DUrole{o}{=} \DUrole{default_value}{\textquotesingle{}\textquotesingle{}}}}{{ $\rightarrow$ {\hyperref[\detokenize{index:Snob2.SnIrrep.str}]{\sphinxcrossref{str}}}}}
\end{fulllineitems}

\index{get\_dim() (SnIrrep method)@\spxentry{get\_dim()}\spxextra{SnIrrep method}}

\begin{fulllineitems}
\phantomsection\label{\detokenize{index:Snob2.SnIrrep.get_dim}}\pysiglinewithargsret{\sphinxbfcode{\sphinxupquote{get\_dim}}}{\emph{\DUrole{n}{self}\DUrole{p}{:} \DUrole{n}{{\hyperref[\detokenize{index:Snob2.SnIrrep}]{\sphinxcrossref{Snob2.SnIrrep}}}}}}{{ $\rightarrow$ \sphinxhref{https://docs.python.org/3/library/functions.html\#int}{int}}}
\sphinxAtStartPar
Return the dimension of the irrep

\end{fulllineitems}

\index{str() (SnIrrep method)@\spxentry{str()}\spxextra{SnIrrep method}}

\begin{fulllineitems}
\phantomsection\label{\detokenize{index:Snob2.SnIrrep.str}}\pysiglinewithargsret{\sphinxbfcode{\sphinxupquote{str}}}{\emph{\DUrole{n}{self}\DUrole{p}{:} \DUrole{n}{{\hyperref[\detokenize{index:Snob2.SnIrrep}]{\sphinxcrossref{Snob2.SnIrrep}}}}}, \emph{\DUrole{n}{indent}\DUrole{p}{:} \DUrole{n}{{\hyperref[\detokenize{index:Snob2.SnIrrep.str}]{\sphinxcrossref{str}}}} \DUrole{o}{=} \DUrole{default_value}{\textquotesingle{}\textquotesingle{}}}}{{ $\rightarrow$ {\hyperref[\detokenize{index:Snob2.SnIrrep.str}]{\sphinxcrossref{str}}}}}
\end{fulllineitems}


\end{fulllineitems}


\begin{DUlineblock}{0em}
\item[] 
\end{DUlineblock}


\chapter{Indices and tables}
\label{\detokenize{index:indices-and-tables}}\begin{itemize}
\item {} 
\sphinxAtStartPar
\DUrole{xref,std,std-ref}{genindex}

\item {} 
\sphinxAtStartPar
\DUrole{xref,std,std-ref}{modindex}

\item {} 
\sphinxAtStartPar
\DUrole{xref,std,std-ref}{search}

\end{itemize}



\renewcommand{\indexname}{Index}
\printindex
\end{document}